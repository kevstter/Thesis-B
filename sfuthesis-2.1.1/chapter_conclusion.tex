\chapter{Conclusion}
In this thesis, we gave details of a framework for developing effective linearly stabilized time stepping methods. As was already known, unconditional stability is required. In our work, we further deduced that the restriction on the parameter $p$, over which we enjoy unconditional stability, must be unbounded. 

Alongside stability, there is the matter of accuracy. As the former dictated the range from which one can select $p$, it was then natural to explore how the discretization error behaves as a function of $p$. The simple procedure of applying the numerical method to the modified test equation and examining the series expansion at $\Delta t = 0$ was recommended. When the coefficient of the leading order error term is a degree two or higher polynomial in $p$, the schemes fared poorly for $p$ large.

We also addressed the feasibility of taking large time step-sizes. The schemes that perform well possess strong damping as $z\to-\infty$. Similar to the previous two qualities, we have shown that this quality is easy to check for any time stepping method.

We have proposed a number of new methods based on IMEX multistep methods and exponential Runge-Kutta methods that perform exceptionally well on at least two of the three aforementioned properties, but it remains to develop higher order variants that excel in all three.

We recommend SBDF2 for its ease of use and its superior damping to CNAB, although CNAB remains a useful alternative as it has a smaller error constant as $\Delta t \to 0$. We have shown that implementing these schemes require no more expertise than applying an implicit method to solve a constant coefficient heat equation. Experiments in image processing, interface motion, and phase separation demonstrate the effectiveness of these schemes.

For problems where the domain is periodic and $p$ can be chosen quite small, one may consider ETDRK2 and ETDRK4. The superior performance of these two methods at large step-sizes more than offsets the higher per step cost when compared to the multistep-based methods. These methods are less conventional but the \textsc{Matlab} code provided in Appendix \ref{A1 sample code} should be sufficient to get an interested user started.

Of the existing linearly stabilized methods, none are competitive with our schemes. 
SBDF1 is only first order accurate. The EIN method, although formally second order accurate, exhibited a reduced order of accuracy in many of our numerical experiments due to a large error constant, and its cost per step is also nearly three times that of our multistep versions. These shortcomings were examined in \cref{LMM,chap:num experiments}, and substantial improvements in accuracy and efficiency were made by using our methods.

A number of questions have been raised throughout this thesis and are worthy of further consideration. 

We have identified three properties critical to effective linearly stabilized schemes.  SBDF2, as we have shown, is a strong candidate, yet it is only second order. The derivation of third and higher order methods excelling in all three remain open. Moreover, as non-periodic boundary conditions are not well-handled by exponential time differencing methods, higher order methods that do not require the matrix exponential would be the most compelling.

A study of the discretization error of linearly stabilized schemes on fully nonlinear problems would also be of interest. While the metrics we supplied are simple to construct and instructive, such a study may provide sharper insight as to when it is advantageous to apply a particular scheme, e.g., ETDRK2. 

Adaptivity also could be investigated. Both the time step-size and the parameter $p$ are candidates for adaptivity in time, although doing so may cost us the efficiency of preprocessing only once. The analysis of test problem \cref{ammc} also suggests adaptivity of $p$ in space may lead to interesting results, although such a procedure likely does not fit within our framework. 

Finally, it would be of interest to assess the performance of our methods against popular algorithms for the solution to stiff nonlinear PDEs. We have already shown that our second order methods outperform the first order method proposed in \cite{schonlieb2011unconditionally} for image inpainting. A more challenging test would be to compete against the split Bregman algorithm \cite{goldstein2009split} implemented in \cite{papafitsoros2014combined,papafitsoros2013combined} for first and second order variational image reconstruction models. 
