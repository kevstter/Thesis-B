\chapter{Conclusion}
In this thesis, we gave details of a framework for developing effective linearly stabilized time stepping methods. As was already known, unconditional stability is required. In our work, we further deduced that the parameter restriction over which we enjoy unconditional stability must be unbounded. 

Alongside stability, there is the matter of accuracy. As the former dictated the range from which we can select the parameter $p$, it was then natural to explore how the discretization error behaves as a function of $p$. We recommended the simple procedure of applying the numerical method to the modified test equation and examining the series expansion at $\Delta t = 0$. When the coefficient of the leading order error term is a degree two or higher polynomial in $p$, the schemes fared poorly for $p$ large.

We also addressed the feasibility of taking large time step-sizes. The schemes that perform well possess strong damping as $z\to-\infty$. We have shown that this quality, as was with the previous two, is easy to check for any time stepping method.

On the matter of developing linearly stabilized schemes with the aforementioned properties, we have proposed a number of new methods based on IMEX multistep methods and exponential Runge-Kutta methods that perform exceptionally on at least two of the three, but it remains to develop higher order variants that excel in all three.

We recommend SBDF2 for its ease of use and its superior damping to CNAB, although CNAB remains a useful alternative as it has a smaller error constant as $\Delta t \to 0$. We have shown that implementing these schemes require no more expertise than applying an implicit method to solve a constant coefficient heat equation. Experiments in image processing, interface motion, and phase separation demonstrate the effectiveness of these schemes.

For problems where the domain is periodic and $p$ can be chosen quite small, one may consider ETDRK2 and ETDRK4. These two schemes offer strong performance at large step-sizes and are equally easy to implement.

Of the existing linearly stabilized methods, none are competitive with our schemes. 
SBDF1 is only first order accurate. The EIN method, although formally second order accurate, exhibited a reduced order of accuracy in many of our numerical experiments due to a large error constant, and its cost per step is also nearly three times that of our multistep versions. 

A number of questions have been raised throughout this thesis and are worthy of further consideration. 

We have identified three properties critical to effective linearly stabilized schemes. SBDF2, as we have shown, is a strong candidate, yet it is only second order. The derivation of third and higher order methods excelling in all three remain open.

A study of the discretization error on fully nonlinear problems would also be of interest. While the metrics we supplied are instructive and simple to compute, such a study may provide sharper insight as to when it is advantageous to apply, e.g., ETDRK2. 

Adaptivity could be investigated. Both the time step-size and the parameter $p$ are candidates, although doing so may cost us the efficiency of preprocessing only once. 

Finally, it would be of interest to assess the performance of our methods against popular algorithms for the solution to stiff nonlinear PDEs. 
