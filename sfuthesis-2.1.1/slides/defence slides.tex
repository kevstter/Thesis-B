\documentclass[hyperref={pdfpagelabels=false}]{beamer}
\usepackage{lmodern}
\usetheme{CambridgeUS}

\usepackage{booktabs}
\usepackage{multirow}
\usepackage{graphicx}

\title[Linearly Stabilized Schemes]{Linearly Stabilized Schemes for the Time Integration of Stiff Nonlinear PDEs}  
\author{Kevin Chow} 
\date{December 9, 2016} 
\begin{document}
%\logo{\includegraphics[scale=0.14]{logo-SF}}
\begin{frame}
\titlepage
\end{frame} 


\begin{frame}
\frametitle{Table of contents}
\tableofcontents
\end{frame} 


\section{Introduction} 
\begin{frame}
\frametitle{Introduction}
\begin{enumerate}
	\item Focus on time stepping for stiff nonlinear PDEs.
	\begin{itemize}
		\item Stability
		\item Accuracy
		\item Efficiency
		\item Simplicity
	\end{itemize}
\end{enumerate}
\end{frame}
\subsection{Example}
\begin{frame}
\frametitle{Example}	
Consider the heat equation, 
\begin{align*}
u_t &= u_{xx}, \quad x \in \Omega, t > 0.
\end{align*}
Discretize in space:
\begin{align*}
U' &= LU, \quad U \in \mathbb{R}^N, t > 0.
\end{align*}
Explicit: $U^{n+1} = G(U^n,U^{n-1},\dots, LU^n, LU^{n-1},\dots)$, but $\Delta t \leq Ch^2$. 

Implicit: $AU^{n+1} = b$; unconditionally stable, but must solve a linear system. 
\end{frame}
\begin{frame}
\frametitle{Example}	
Now compare with 
\begin{align*}
u_t &= \frac{u_{xx}}{1 + u_x^2} - \frac{1}{u}, \quad x \in \Omega, t > 0.
\end{align*}
and 
\begin{align*} 
U' &= F(U), \quad U \in \mathbb{R}^N, t > 0.
\end{align*}
Explicit: $U^{n+1} = G(U^n,U^{n-1},\dots, F(U^n), F(U^{n-1}),\dots)$, but $\Delta t \leq Ch^2$. 

Implicit: $AU^{n+1} = b(U^{n+1})$; unconditionally stable, but must solve a nonlinear system because nonlinearity is in the stiff term. 
\end{frame}
\begin{frame}
	\frametitle{Example}
	Comparing side-by-side:
	\begin{columns}
\begin{column}{0.48\textwidth}
	\begin{align*} 
		u_t = u_{xx}, \quad x\in \Omega, t > 0,
	\end{align*}
Explicit: $\Delta t \leq Ch^2$

Implicit: unconditionally stable; solution to linear system
\end{column}
\begin{column}{0.48\textwidth}
\begin{align*} 
	u_t = \frac{u_{xx}}{1 + u_x^2} - \frac{1}{u}, \quad x\in \Omega, t > 0,
\end{align*}
Explicit: $\Delta t \leq Ch^2$

Implicit: unconditionally stable; solution to nonlinear system
\end{column}
	\end{columns}
\begin{alertblock}{Summary: What We Like}
	Explicit: simple; handles nonlinear terms with no added difficulty.
	
	Implicit: large time steps
\end{alertblock}	
\end{frame}
\begin{frame}
	\frametitle{Example}
Modify the equation,
\begin{align*}
u_t &= \frac{u_{xx}}{1 + u_x^2} - \frac{1}{u} - u_{xx} + u_{xx}, \quad x\in \Omega, t>0, 
\\
&\downarrow 
\\
U' &= F(U) - LU + LU, \quad U\in \mathbb{R}^N, t > 0,
\end{align*}
and use implicit-explicit time stepping, e.g.
\begin{align*}
\frac{U^{n+1} - U^n}{\Delta t} = F(U^n) - LU^n + LU^{n+1}.
\end{align*}
\end{frame}

\section{Linear Stability} 
\begin{frame}
\frametitle{Linear Stability}
More generally, from $U'=F(U)$, we can modify as 
\begin{align*}
 U' = \underbrace{F(U) - pLU}_{(\star)} + pLU, \quad p > 0,
\end{align*}
and apply a time stepping scheme that treats $(\star)$ explicitly.

Key question: Is this unconditionally stable? 	
\end{frame}
\subsection{Test equation}
\begin{frame}
	\frametitle{Scalar test equation}
\begin{columns}
	\begin{column}{0.46\textwidth}
Standard case:
\begin{align*} 
U' = F(U)
\end{align*} 

Linearize $\to$ Diagonalize $\to$ Test equation: 
\begin{align*} 
u' = \lambda u
\end{align*} 
	\end{column}
	\begin{column}{0.46\textwidth}
With linear modification: 
\begin{align*} 
U' = F(U) - pLU + pLU
\end{align*}

Linearize $\to$ Diagonalize $\to$ Test equation: 
\begin{align*}
u' &= \lambda u - p\lambda u + p\lambda u 
\\
&= (1-p)\lambda u + p\lambda u 
\end{align*}
		\end{column}
\end{columns}
Apply time stepping method: 
\begin{align*} 
u^{n+1} = \xi(\lambda\Delta t) u^n
\end{align*} 

Stability constraint: 
\begin{align*}
\left\vert \xi(\lambda\Delta t)\right\vert \leq 1
\end{align*}
\end{frame}
\subsection{IMEX Euler}
\begin{frame}
	\frametitle{Implicit-explicit Euler} 
Applied to the test equation $u' = (1-p)\lambda u + p\lambda u$, yields 
\begin{align*}
\frac{u^{n+1} - u^n}{\Delta t} = (1-p)\lambda u^n + p\lambda u^{n+1}.
\end{align*}
The amplification factor is 
\begin{align*}
\xi_1(\lambda\Delta t) 
= \frac{1 + (1-p)\lambda\Delta t}{1 - p\lambda\Delta t}. 
\end{align*}
Impose unconditional stability: 
\begin{align*}
\left\vert \xi_1(\lambda\Delta t)\right \vert \leq 1  
\text{ for all }\lambda \Delta t < 0 \implies p \geq 1/2.
\end{align*}
\end{frame}
\subsection{EIN} 
\begin{frame}
	\frametitle{Explicit-implicit-null (EIN)} 
Use Richardson extrapolation to get second order. The amplification factor is  	
\begin{align*}
\xi_{EIN}(\lambda\Delta t) 
= 2\xi^2_1(\lambda\Delta t/2) - \xi_1(\lambda\Delta t).
\end{align*}
and 
\begin{align*}
\left\vert \xi_{EIN}(\lambda\Delta t) \right\vert \leq 1 \text{ for all } \lambda\Delta t < 0 \implies p \geq 2/3.
\end{align*}
\end{frame}
\subsection{IMEX Multistep}
\begin{frame}
	\frametitle{Implicit-explicit multistep methods}
An alternative for second and higher order methods: IMEX multistep methods.	
\begin{table}[htb!]
	\centering  
%	\caption{Parameter ranges for select IMEX LMM.}
	\begin{tabular}{lllr}
		\toprule
		Order && Method & $p\in$
		\\ \midrule 
%		1 && IMEX-Euler & $(1/2,\infty)$ 
%		\\ \cmidrule{1-3}
		\multirow{4}{*}{2} && SBDF2 & $[3/4,\infty)$
		\\
		&& CNAB & $[1,\infty)$ 
		\\
		&& mCNAB & $[8/9,\infty)$ 
		\\
		&& CNLF & $[1/2,\infty)$
		\\ \cmidrule{1-3}
		3 && SBDF3 & $[7/8,2]$ 
		\\ \cmidrule{1-3}
		4 && SBDF4 & $[11/12,5/4]$
		\\
		\bottomrule
	\end{tabular}
\end{table}
\end{frame}

\section{Comparing the Methods} 
\begin{frame}
	\frametitle{Comparing the methods}
Do the methods work as advertised? 	

Examine this with two test problems,
\begin{align*}
u_t = \frac{u_{xx}}{1 + u_x^2} - \frac{1}{u}, 
\end{align*}
and 
\begin{align*}
u_t = \Delta(u^5).
\end{align*}
\end{frame}

\subsection{Test Problem 1} 
\begin{frame}
	\frametitle{Test Problem 1} 
First test problem: 
\begin{align*}
	u_t = \frac{u_{xx}}{1 + u_x^2} - \frac{1}{u}, 
	\quad 0 < x < 10,\quad t > 0, 
\end{align*}
with initial condition 
\begin{align*}
	u(x,0) = 1 + 0.10\sin\left(\frac{\pi}{5}x \right),
\end{align*}
and boundary conditions $u(0,t) = 1 = u(10,t)$.
\end{frame}
\begin{frame}
	\frametitle{Numerical convergence test}
	\begin{figure}[t]
		\centering
		%\includegraphics[width=0.65\textwidth]{}
		\caption{Numerical convergence of linearly stabilized schemes.}
	\end{figure}
	Spatial discretization: Uniform grid, centred differences, N=2048.
	
	Reference solution: Explicit third order Runge-Kutta, $\Delta t = 1.46\times 10^{-5}$.	
\end{frame}
\begin{frame}
	\frametitle{Failure of SBDF3 and SBDF4}

\end{frame}

\subsection{Test Problem 2}
\begin{frame}
	\frametitle{Test Problem 2} 
Second test problem: 
\begin{align*}
u_t = \Delta(u^5), 
\quad (x,y) \in [0,1]^2, \quad t > 0, 
\end{align*}
with initial and boundary conditions set such that the exact solution is 
\begin{align*}
u(x,y,t) = \left(\frac{4}{5}(2t+x+y) \right)^{1/4}.
\end{align*}
\end{frame}
\begin{frame}
	\frametitle{Numerical convergence test}
	\begin{figure}[t]
		\centering
		%\includegraphics[width=0.65\textwidth]{}
		\caption{Numerical convergence of linearly stabilized schemes.}
	\end{figure}
	Spatial discretization: second order centred differences; N=2048.
	
	Reference solution: explicit third order Runge-Kutta, $\Delta t = 1.46\times 10^{-5}$.
\end{frame}
\begin{frame}
	\frametitle{Error constant}
\end{frame}


\subsection{Tables}
\begin{frame}
\frametitle{Tables}
\begin{tabular}{|c|c|c|}
\hline
\textbf{Date} & \textbf{Instructor} & \textbf{Title} \\
\hline
WS 04/05 & Sascha Frank & First steps with  \LaTeX  \\
\hline
SS 05 & Sascha Frank & \LaTeX \ Course serial \\
\hline
\end{tabular}
\end{frame}


\begin{frame}
\frametitle{Tables with pause}
\begin{tabular}{c c c}
A & B & C \\ 
\pause 
1 & 2 & 3 \\  
\pause 
A & B & C \\ 
\end{tabular} 
\end{frame}



\section{Exponential Runge-Kutta Methods} 
\subsection{Tables}
\begin{frame}
	\frametitle{Tables}
	\begin{tabular}{|c|c|c|}
		\hline
		\textbf{Date} & \textbf{Instructor} & \textbf{Title} \\
		\hline
		WS 04/05 & Sascha Frank & First steps with  \LaTeX  \\
		\hline
		SS 05 & Sascha Frank & \LaTeX \ Course serial \\
		\hline
	\end{tabular}
\end{frame}


\begin{frame}
	\frametitle{Tables with pause}
	\begin{tabular}{c c c}
		A & B & C \\ 
		\pause 
		1 & 2 & 3 \\  
		\pause 
		A & B & C \\ 
	\end{tabular} 
\end{frame}


\section{Numerical Experiments}
\subsection{blocs}
\begin{frame}
\frametitle{blocs}

\begin{block}{title of the bloc}
bloc text
\end{block}

\begin{exampleblock}{title of the bloc}
bloc text
\end{exampleblock}


\begin{alertblock}{title of the bloc}
bloc text
\end{alertblock}
\end{frame}


\begin{frame}\frametitle{lists with single pauses}
	\begin{itemize}
		\item Introduction to  \LaTeX{}  \pause 
		\item Course 2 \pause 
		\item Termpapers and presentations with \LaTeX{}  \pause 
		\item Beamer class
	\end{itemize} 
\end{frame}

\begin{frame}\frametitle{lists with pause}
	\begin{itemize}[<+->]
		\item Introduction to  \LaTeX{}  
		\item Course 2
		\item Termpapers and presentations with \LaTeX{}  
		\item Beamer class
	\end{itemize} 
\end{frame}

\subsection{Lists II}

\begin{frame}\frametitle{numbered lists}
	\begin{enumerate}
		\item Introduction to  \LaTeX{}   
		\item Course 2 
		\item Termpapers and presentations with \LaTeX{}  
		\item Beamer class
	\end{enumerate}
\end{frame}

\begin{frame}
	\frametitle{numbered lists with single pauses}
	\begin{enumerate}
		\item Introduction to  \LaTeX{}  \pause 
		\item Course 2 \pause 
		\item Termpapers and presentations with \LaTeX{}  \pause 
		\item Beamer class
	\end{enumerate}
\end{frame}

\begin{frame}
	\frametitle{numbered lists with pause}
	\begin{enumerate}[<+->]
		\item Introduction to  \LaTeX{}  
		\item Course 2
		\item Termpapers and presentations with \LaTeX{}  
		\item Beamer class
	\end{enumerate}
\end{frame}


\section{Conclusion}


\end{document}
