%\PassOptionsToPackage{demo}{graphicx}
\documentclass[hyperref={pdfpagelabels=false}]{beamer}
\usepackage{lmodern}
\usetheme{CambridgeUS}

\usepackage{booktabs}
\usepackage{multirow}

\title[Linearly Stabilized Schemes]{Linearly Stabilized Schemes for the Time Integration of Stiff Nonlinear PDEs}  
\author[Kevin Chow]{Kevin Chow} 
\date{December 9, 2016} 

\begin{document}
\begin{frame}
\titlepage
\end{frame} 

\begin{frame}
\frametitle{Table of contents}
\tableofcontents
\end{frame} 

\section{Introduction} 
\begin{frame}
	\frametitle{Introduction}
\begin{itemize}
	\item Focus on time stepping for stiff nonlinear PDEs.
	\begin{itemize}
		\item Stability
		\item Accuracy
		\item Simplicity
		\item Efficiency
	\end{itemize}
\end{itemize}
\end{frame}

\subsection{Example}
\begin{frame}
	\frametitle{Example}
Consider the heat equation, 
	\begin{align*}
		u_t &= u_{xx}, \quad x \in \Omega, \quad t > 0.
	\end{align*}
Discretize in space:
	\begin{align*}
		U' &= LU, \quad U \in \mathbb{R}^N, \quad t > 0.
	\end{align*}
Explicit: $U^{n+1} = G(U^n,U^{n-1},\dots, LU^n, LU^{n-1},\dots)$, but $\Delta t \leq Ch^2$. 

Implicit: $AU^{n+1} = b$; unconditionally stable, but must solve a linear system. 
\end{frame}

\begin{frame}
	\frametitle{Example}
Now compare with 
	\begin{align*}
		u_t &= \frac{u_{xx}}{1 + u_x^2} - \frac{1}{u}, \quad x \in \Omega, \quad t > 0.
	\end{align*}
and 
	\begin{align*} 
		U' &= F(U), \quad U \in \mathbb{R}^N, \quad t > 0.
	\end{align*}
Explicit: $U^{n+1} = G(U^n,U^{n-1},\dots, F(U^n), F(U^{n-1}),\dots)$, but $\Delta t \leq Ch^2$. 

Implicit: $AU^{n+1} = b(U^{n+1})$; unconditionally stable, but must solve a nonlinear system because nonlinearity is in the stiff term. 
\end{frame}

\begin{frame}
	\frametitle{Example}
Comparing side-by-side:
	\begin{columns}
\begin{column}{0.48\textwidth}
	\begin{align*} 
		u_t = u_{xx}, \quad x\in \Omega, t > 0,
	\end{align*}
Explicit: $\Delta t \leq Ch^2$

Implicit: unconditionally stable; solution to linear system
\end{column}
\begin{column}{0.48\textwidth}
	\begin{align*} 
		u_t = \frac{u_{xx}}{1 + u_x^2} - \frac{1}{u}, \quad x\in \Omega, t > 0,
	\end{align*}
Explicit: $\Delta t \leq Ch^2$

Implicit: unconditionally stable; solution to nonlinear system
\end{column}
	\end{columns}
	\begin{alertblock}{Summary: What We Like}
		Explicit: simple; handles nonlinear terms with no added difficulty.

		Implicit: large time steps
	\end{alertblock}
\end{frame}

\begin{frame}
	\frametitle{Example}
Modify the equation,
	\begin{align*}
		u_t &= \frac{u_{xx}}{1 + u_x^2} - \frac{1}{u} - u_{xx} + u_{xx}, \quad x\in \Omega, \quad t>0, 
	\end{align*}
and discretize in space,
	\begin{align*}
		U' &= F(U) - LU + LU, \quad U\in \mathbb{R}^N,\quad t > 0.
	\end{align*}
Use implicit-explicit time stepping, e.g.
	\begin{align*}
		\frac{U^{n+1} - U^n}{\Delta t} = F(U^n) - LU^n + LU^{n+1}.
	\end{align*}
\end{frame}

\section{Linear Stability Analysis} 
\begin{frame}
	\frametitle{Linear Stability}
More generally, from $U'=F(U)$, we can modify as 
	\begin{align*}
 		U' = \underbrace{F(U) - pLU}_{(\star)} + pLU, \quad p > 0,
	\end{align*}
and apply a time stepping scheme that treats $(\star)$ explicitly.

\vspace{5pt}
\textbf{Is this unconditionally stable?}
\end{frame}

\subsection{Test equation}
\begin{frame}
	\frametitle{Scalar test equation}
	\begin{columns}
\begin{column}{0.46\textwidth}
	Standard case:
	\begin{align*} 
		U' = F(U)
	\end{align*} 

Linearize $\to$ Diagonalize $\to$ Test equation: 
	\begin{align*} 
		u' = \lambda u
	\end{align*} 
\end{column}
\begin{column}{0.46\textwidth}
	With linear modification: 
	\begin{align*} 
		U' = F(U) - pLU + pLU
	\end{align*}

Linearize $\to$ Diagonalize $\to$ Test equation: 
	\begin{align*}
		u' &= \lambda u - p\lambda u + p\lambda u 
\\
		&= (1-p)\lambda u + p\lambda u 
	\end{align*}
\end{column}
	\end{columns}
Apply time stepping method: 
	\begin{align*} 
		u^{n+1} = \xi(\lambda\Delta t) u^n.
	\end{align*} 

Unconditional stability: 
	\begin{align*}
		\left\vert \xi(\lambda\Delta t)\right\vert \leq 1
		\quad\text{for all}\quad \lambda\Delta t < 0.
	\end{align*}
\end{frame}

\subsection{IMEX Euler}
\begin{frame}
	\frametitle{Implicit-explicit Euler} 
Applied to the test equation, $u' = (1-p)\lambda u + p\lambda u$, yields 
	\begin{align*}
		\frac{u^{n+1} - u^n}{\Delta t} = (1-p)\lambda u^n + p\lambda u^{n+1}.
	\end{align*}
The amplification factor is 
	\begin{align*}
		\xi_1(\lambda\Delta t) 
		= \frac{1 + (1-p)\lambda\Delta t}{1 - p\lambda\Delta t}. 
	\end{align*}
Impose unconditional stability: 
	\begin{align*}
		\left\vert \xi_1(\lambda\Delta t)\right \vert \leq 1  
		\text{ for all }\lambda \Delta t < 0 \iff p \geq 1/2.
	\end{align*}
\end{frame}

\subsection{Explicit-Implicit-Null (EIN)} 
\begin{frame}
	\frametitle{Explicit-implicit-null (EIN)} 
Duchemin and Eggers (2014) use Richardson extrapolation to get second order. The amplification factor is  
	\begin{align*}
		\xi_{EIN}(\lambda\Delta t) 
		= 2\xi^2_1(\lambda\Delta t/2) - \xi_1(\lambda\Delta t).
	\end{align*}
and 
	\begin{align*}
		\left\vert \xi_{EIN}(\lambda\Delta t) \right\vert \leq 1 \text{ for all } \lambda\Delta t < 0 \iff p \geq 2/3.
	\end{align*}
\end{frame}

\subsection{IMEX Multistep Methods}
\begin{frame}
	\frametitle{Implicit-explicit multistep methods}
An alternative for second and higher order methods: IMEX multistep methods.
\vspace{-10pt}
\begin{table}[htb!]
	\centering  
	\caption{Parameter restriction for select IMEX methods.}
	\begin{tabular}{lllr}
		\toprule
		Order && Method & $p\in$
		\\ \midrule 
		1 && IMEX-Euler & $[1/2,\infty)$ 
		\\ \cmidrule{1-3}
		\multirow{4}{*}{2} && SBDF2 & $[3/4,\infty)$
		\\
		&& CNAB & $[1,\infty)$ 
		\\
		&& mCNAB & $[8/9,\infty)$ 
		\\
		&& CNLF & $[1/2,\infty)$
		\\ \cmidrule{1-3}
		3 && SBDF3 & $[7/8,2]$ 
		\\ \cmidrule{1-3}
		4 && SBDF4 & $[11/12,5/4]$
		\\
		\bottomrule
	\end{tabular}
\end{table}
\end{frame}

\section{Understanding the Methods: 3 Key Properties} 
\begin{frame}
	\frametitle{Understanding the methods: 3 key properties}
Do the methods work as advertised? 	

Examine this with two test problems,
	\begin{align*}
		u_t = \frac{u_{xx}}{1 + u_x^2} - \frac{1}{u}, 
	\end{align*}
and 
	\begin{align*}
		u_t = \Delta(u^5).
	\end{align*}
\end{frame}

\subsection{Test Problem 1} 
\begin{frame}
	\frametitle{Test Problem 1} 
First test problem: 
	\begin{align*}
		u_t = \frac{u_{xx}}{1 + u_x^2} - \frac{1}{u}, 
		\quad 0 < x < 10,\quad t > 0, 
	\end{align*}
with initial condition 
	\begin{align*}
		u(x,0) = 1 + 0.10\sin\left(\frac{\pi}{5}x \right),
	\end{align*}
and boundary conditions $u(0,t) = 1 = u(10,t)$.

Stabilized as 
\begin{align*}
u_t = \frac{u_{xx}}{1 + u_x^2} - \frac{1}{u} - pu_{xx} + pu_{xx}.
\end{align*}
\end{frame}

\begin{frame}
	\frametitle{Numerical convergence test}
	\begin{figure}[t]
		\centering
		\includegraphics[width=0.55\textwidth]{lmm_conv_ammc}
		\caption{Numerical convergence of linearly stabilized schemes at time $T=0.35$. Compared against a reference solution generated using an explicit 3rd order RK with $\Delta t = 1.46\times 10^{-5}$. Stabilized by adding and subtracting $pu_{xx}$.}
	\end{figure}
\end{frame}

\begin{frame}
	\frametitle{Selecting $p$ for more general problems}
How did we choose $p$? Consider 
	\begin{align*}
		u' = \lambda u - p\lambda u + p\lambda u
	\end{align*}
and 
	\begin{align*}
		U' = F(U) - pLU + pLU.
	\end{align*}
With the test equation, we derived a restriction on $p$. More generally, the restriction applies to $p\lambda_L / \lambda_F$. For test problem 1 with centred differences, we find 
	\begin{align*}
		\frac{p\lambda_L}{\lambda_F} 
		\approx p(1 + (D_1 \bar u^n_j)^2).
	\end{align*}
\end{frame}

\begin{frame}
	\frametitle{Rejecting methods with bounded $p$-parameter restrictions} 
The selection of $p$ for SBDF3 is dictated by 
	\begin{align*}
		\max_{1\leq j\leq N} \frac{7}{8} \frac{1}{1 + (D_1 \bar u^n_j)^2} 
		\leq p
		\leq \min_{1\leq j\leq N} \frac{2}{1 + (D_1 \bar u^n_j)^2}.
	\end{align*}
	\begin{figure}
		\centering 
\begin{minipage}{0.45\textwidth} 
	\centering
        \includegraphics[width=0.90\textwidth]{ammc_sol}
	\caption{Numerical solution to test problem 1 .}
\end{minipage}
\begin{minipage}{0.45\textwidth} 
	\centering
	\includegraphics[width=0.90\textwidth]{sbdf3_instabilities}
	\caption{Development of instabilities using SBDF3, $p=1.625$.}
\end{minipage} 
	\end{figure}
\end{frame}

\subsection{Test Problem 2}
\begin{frame}
	\frametitle{Test Problem 2} 
Second test problem: 
	\begin{align*}
		u_t = \Delta(u^5), 
		\quad (x,y) \in [0,1]^2, \quad t > 0, 
	\end{align*}
with initial and boundary conditions set such that the exact solution is 
	\begin{align*}
		u(x,y,t) = \left(\frac{4}{5}(2t+x+y) \right)^{1/4}.
	\end{align*}
	
Stabilize with $p\Delta u$; $p\lambda_L/\lambda_F \approx p/(8(1+t))$.
\end{frame}

\begin{frame}
	\frametitle{Numerical convergence test}
	\begin{figure}[t]
		\centering
		\includegraphics[width=0.55\textwidth]{nl5_conv}
		\caption{Numerical convergence of linearly stabilized schemes at time $T=0.40$. Compared against a reference solution generated using an explicit 3rd order RK with $\Delta t = 6.25\times 10^{-6}$. Stabilized using $p\Delta u$; $p\lambda_L / \lambda_F \approx p/(8(1+t))$.}
	\end{figure} 
\end{frame}

\begin{frame}
	\frametitle{Error coefficient and dependence on $p$}
Discretizing $u' = (1- p)\lambda u + p\lambda u$, we observe that the discretization error grows with $p$.

\textbf{How does the error behave as we increase $p$?}

Examine the coefficient of the leading order error term.
\vspace{-11pt}
\begin{table}[b]
	\centering 
	\caption{Coefficient of leading order error term as applied to the test equation.}
\renewcommand{\arraystretch}{1.10}
	\begin{tabular}{ll} 
		\toprule 
Method & Coefficient 
\\ \midrule 
EIN & $\frac{1}{2}(p-p^2)$ 
\\
SBDF2 & $\frac{2}{3}p - \frac{5}{18}$
\\ 
CNAB & $\frac{1}{2}p-\frac{1}{4}$
\\ 
CNLF & $p-\frac{1}{6}$
\\ \bottomrule
		\end{tabular}
	\end{table}
\end{frame}


\begin{frame}
	\frametitle{Numerical convergence test}
	\begin{figure}[t]
		\centering
		\includegraphics[width=0.55\textwidth]{nl5_conv}
		\caption{Numerical convergence of linearly stabilized schemes at time $T=0.40$. Compared against a reference solution generated using an explicit 3rd order RK with $\Delta t = 6.25\times 10^{-6}$. Stabilized using $p\Delta u$; $p\lambda_L / \lambda_F \approx p/(8(1+t))$.}
	\end{figure} 
\end{frame}

\begin{frame}
	\frametitle{Amplification factor as $\Delta t\to\infty$}
Methods with small amplication factor as $\Delta t \to \infty$ perform well as large step-sizes. 
\vspace{-5pt}
\begin{figure}
	\centering 
	\includegraphics[width=0.55\textwidth]{damp_at_inf_3}
	\caption{Amplification factor as $\Delta t \to \infty$.}
\end{figure}
\end{frame}

\section{Numerical Experiments}
\begin{frame}
        \frametitle{Numerical Experiments} 
We consider two classes of problems to demonstrate the effectiveness of our new methods: 
\begin{enumerate}
        \item Image inpainting.
	\item Mean curvature motion.
\end{enumerate}

\end{frame}

\subsection{Image Inpainting} 
\begin{frame}
	\frametitle{Image Inpainting} 
	In Sch\"{o}nlieb and Bertozzi (2011), the authors proposed the fourth order inpainting model 
	\begin{align*}
	u_t = -\Delta \nabla \cdot\left( \frac{\nabla u}{\sqrt{\left\vert \nabla u \right\vert^2 + \epsilon^2}} \right) + \lambda_0 (u_0 - u),
	\end{align*}
	and numerical solution by the first order accurate method 
	\begin{align*}
	\frac{u^{n+1} - u^n}{\Delta t} 
	=  -\Delta \nabla \cdot\left( \frac{\nabla u^n}{\sqrt{\left\vert \nabla u^n \right\vert^2 + \epsilon^2}} \right) &+ \lambda_0 (u_0 - u^n) 
\\ +p_1\Delta^2u^n - p_1\Delta^2 u^{n+1} 
	&+ p_0\lambda u^n  - p_0\lambda u^{n+1}.
	\end{align*}
\end{frame}
\begin{frame}
	\frametitle{Image inpainting} 
	\begin{figure}
		\centering
\begin{minipage}{0.45\textwidth}
\centering
	\includegraphics[width=0.75\textwidth]{bullfinch_and_fox}
\end{minipage}
\begin{minipage}{0.45\textwidth}
\centering
        \includegraphics[width=0.75\textwidth]{bvhneg_bullfinch}
\end{minipage}
\caption{TV-H$^{-1}$ image restoration.} 
	\end{figure}
\vspace{-15pt}
	\begin{table}[b]
		\centering 
	\caption{Iteration counts for TV-H$^{-1}$ image restoration.}
		\begin{tabular}{lll}
			\toprule 
			& $\Delta t$ & Iterations 
			\\ \midrule 
		IMEX Euler & 0.30 & 1002 
		\\ 
		SBDF2 & 0.54 & 401 
		\\ 
		CNAB & 0.64 & 347
		\\ \bottomrule
		\end{tabular}
	\end{table} 
\end{frame}

\subsection{Motion by Mean Curvature}
\begin{frame}
	\frametitle{Motion by mean curvature}
We evolve the level set equation for motion by mean curvature, 
\begin{align*}
u_t 
= \kappa \left\vert \nabla u \right\vert 
= \left\vert \nabla u \right\vert \nabla \cdot \left(\frac{\nabla u}{\left\vert \nabla u \right\vert} 
\right)
\end{align*}
on an initial dumbbell-shaped curve in 3D. We solve to time $T=0.75$ on a $256\times 128\times 128$ periodic grid over $(-5,5) \times (-2,2) \times (-2,2)$.
%$\left\{(x,y,z \mid \left\vert x \right\vert < 5, \left\vert y \right\vert < 2, \left\vert z \right\vert < 2)\right\}$.
\end{frame}
\begin{frame}
	\frametitle{Motion by mean curvature} 
\begin{figure}[htb!]
        \centering
\begin{minipage}{0.30\textwidth}
        \includegraphics[width=0.96\textwidth]{dumbbell_3d0}
\end{minipage}%
\begin{minipage}{0.30\textwidth}
        \includegraphics[width=0.96\textwidth]{dumbbell_3d01}
\end{minipage}%
\begin{minipage}{0.30\textwidth}
        \includegraphics[width=0.96\textwidth]{dumbbell_3d03}
\end{minipage}
\begin{minipage}{0.30\textwidth}
        \includegraphics[width=0.96\textwidth]{dumbbell_3d0525}
\end{minipage}%
\begin{minipage}{0.30\textwidth}
        \includegraphics[width=0.96\textwidth]{dumbbell_3d055}
\end{minipage}%
\begin{minipage}{0.30\textwidth}
        \includegraphics[width=0.96\textwidth]{dumbbell_3d075}
\end{minipage}
\caption[Mean curvature flow of a dumbbell-shaped curve in 3D]{Mean curvature flow of a dumbbell-shaped curve in 3D. From the left to right, top to bottom, the plots show the evolution at times $t=0$, $0.10$, $0.30$, $0.525$, $0.55$, $0.75$.}
\end{figure}
\end{frame}
\begin{frame}
        \frametitle{Motion by mean curvature}
On a machine with an Intel\textsuperscript{\textregistered}Core\textsuperscript{\texttrademark}i5-4570 CPU@3.20GHz running \textsc{Matlab} 2014b: 
\begin{itemize}
	\item Forward Euler: 3000 time steps $\to$ over 28 minutes.
	
	\item SBDF2: 75 time steps $\to$ under 100 seconds.
\end{itemize}
\end{frame}

\section{Conclusion}
\begin{frame}
	\frametitle{Conclusion}
	Contributions of this thesis mainly fall into two categories:
\begin{enumerate}
	\item Further developed linearly stabilized schemes.
	\begin{itemize}
		\item Outlined a framework for developing new methods of this type. 
		
		\item Identified properties necessary for effective schemes. 
	\end{itemize}
	
	\item Proposed new methods that outperform existing ones. 
	\begin{itemize}
		\item IMEX multistep methods and exponential time differencing methods. 
	\end{itemize}
\end{enumerate}	

Future work: 
\begin{enumerate}
	\item Development higher order methods without the deficiencies exhibited by ETD methods.
	
	\item A comparison with popular algorithms for nonlinear stiff PDEs, particularly for image inpainting.
\end{enumerate}
\end{frame}


\end{document}
