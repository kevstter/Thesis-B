\chapter{IMEX Linear Multistep Methods}
For equations whose righthand side comprise of a stiff linear component and a nonstiff nonlinear part, a popular class of methods to apply are the implicit-explicit linear multistep methods\footnote{We will refer to these simply as IMEX methods.}. The simplest of these is the semi-implicit Euler -- forward Euler to the nonlinear component and backward Euler to the linear, stiff component -- a scheme that we reviewed in Chapter 2. 

In this chapter, we investigate the use of IMEX methods within the linear stabilization framework. Implicit solution of the added linear term provides the needed stability, and the remaining terms, including the stiff nonlinear term, are solved explicitly.

\section{IMEX Formulas}
In \cite{ascher1995implicit}, IMEX schemes up to order four are investigated and a select number are singled out for their extensive use in the literature or for desired properties such as strong high frequency damping. As we are familiar with the first order variant, we begin by listing the second order methods of interest. These, and the higher order variants, are presented as applied to the ODE 
\begin{align*}
u' = f + g, 
\end{align*}
where $f$ we identify as the nonlinear/nonstiff component and $g$ the stiff linear component. 
\\ \noindent
\textbf{Second order methods}\\
CNAB:
\begin{align}
\frac{u^{n+1}-u^n}{\Delta t} 
= \frac{3}{2} f^n - \frac{1}{2}f^{n-1} 
+ \frac{1}{2}(g^{n+1} + g^n), 
\label{cnab}
\end{align}\\
mCNAB:
\begin{align}
\frac{u^{n+1}-u^n}{\Delta t} 
= \frac{3}{2}f^n - \frac{1}{2} f^{n-1}
+ \frac{9}{16}g^{n+1} 
+ \frac{3}{8}g^n
+ \frac{1}{16}g^{n-1},
\label{mcnab}
\end{align}
\\
CNLF:
\begin{align}
\frac{u^{n+1}-u^{n-1}}{2\Delta t}
= f^n + \frac{1}{2}(g^{n+1} + g^{n-1}),
\label{cnlf}
\end{align} \\
SBDF2:
\begin{align}
\frac{3u^{n+1}-4u^n+u^{n-1}}{2\Delta t} 
= 2f^n - f^{n-1} + g^{n+1}.
\label{sbdf2}
\end{align}

\noindent
\textbf{Third order methods}\\
SBDF3:
\begin{align}
\frac{1}{\Delta t}\left(\frac{11}{6}u^{n+1} - 3u^n + \frac{3}{2}u^{n-1} - \frac{1}{3}u^{n-2} \right) 
= 3f^n - 3f^{n-1} + f^{n-2} + g^{n+1}.
\label{sbdf3}
\end{align}

\noindent
\textbf{Fourth order methods}\\
SBDF4:
\begin{align}
\frac{1}{\Delta t}\left(\frac{25}{12}u^{n+1} - 4u^n + 3u^{n-1} - \frac{4}{3}u^{n-2} + \frac{1}{4}u^{n-3} \right) 
= 4f^n - 6f^{n-1} + 4f^{n-2} -f^{n-3} + g^{n+1}.
\label{sbdf4}
\end{align}

In the next section, we will apply these IMEX schemes to the modified test equation to determine for each scheme the range of $p$ suitable for linear stabilization.

\section{Analysis of the Amplification Polynomials}
Let us restate here the modified test equation and the basic assumptions we make.
The modified test equation is 
\begin{align*}
        u' = (1-p)\lambda u + p\lambda u,
\quad\text{where}\quad \lambda < 0\text{, } p>0.
\label{mtee}
\end{align*}
The goal is, for each IMEX scheme, to identify the constraint on the parameter $p$ such that when satisfied, we may freely choose the time step-size without being subject to a stability restriction.

\subsection{Schur and von Neumann polynomials}
The resulting polynomials from applying an $n$th order IMEX scheme to the modified test equation will be a degree $n$ polynomial in the amplification factor, $\xi$. The tool of choice for analyzing these amplification polynomials is the theory of von Neumann polynomials \cite[Chapter 4]{strikwerda2004finite}. Below we give the relevant definitions and theorems.

\begin{definition}
	The polynomial $\phi$ is a Schur polynomial if all its roots, $r_q$, satisfy $\abs{r_q} < 1$.
\end{definition}
\begin{definition}
        The polynomial $\phi$ is a von Neumann polynomial if all its roots, $r_q$, satisfy $\abs{r_q}\leq 1$.
\end{definition}
\begin{definition}
	The polynomial $\phi$ is a simple von Neumann polynomial is $\phi$ is a von Neumann polynomial and its roots on the unit circle are simple roots.
\end{definition}
\begin{definition}
	For any polynomial $\phi(\xi) = \sum^n_{j=0} a_j\xi^j$, we define the polynomial $\phi^*$ by $\phi^*(\xi) = \sum^n_{j=0} a^*_{n-j} \xi^j$, where $^*$ on the coefficient denotes the complex conjugate.
\label{defn:conj}
\end{definition}
\begin{definition}
	For any polynomial $\phi_n(\xi) = \sum^n_{j=0} a_j\xi^j$, we define recursively the polynomial $\phi_{n-1}$ by
	\begin{align}
	\phi_{n-1}(\xi) = \frac{\phi_n^*(0)\phi_n(\xi) - \phi_n(0)\phi_n^*(\xi)}{\xi}.
	\end{align} 
\label{defn:recurse}
\end{definition}
\begin{theorem}
	$\phi_n$ is a simple von Neumann polynomial if and only if either 
	\begin{enumerate}[label=(\alph{*})]
		\item $\abs{\phi_n(0)} < \abs{\phi^*_n(0)}$ and $\phi_{n-1}$ is a simple von Neumann polynomial or
		
		\item $\phi_{n-1}$ is identically zero and $\phi'_{n}$ is a Schur polynomial.
	\end{enumerate}
\label{thm:simple vN}
\end{theorem}

\subsection{Amplification polynomials of second order IMEX schemes}
We start by applying CNAB \eqref{cnab} to the modified test equation \eqref{mte}. Combined with the anzatz $u^n = \xi^n$ and setting $z=\lambda\Delta t$, we get the amplification polynomial
\begin{align}
\Phi_2(\xi) 
= \left(1 - \frac{1}{2}zp\right)\xi^2
- \left(1 + z\left(\frac{3}{2}-p \right)\right)\xi + \frac{1}{2}z(1-p).
\label{amp_cnab}
\end{align}
The next series of steps will show that for all $z<0$ (i.e.\ $\lambda < 0$,) \eqref{amp_cnab} is a simple von Neumann polynomial if and only if $p\geq 1$. We do so by showing that \cref{thm:simple vN} holds for $\Phi_2$.

We first give $\Phi_2^*$ and $\Phi_1$ as defined by the processes in \cref{defn:conj,defn:recurse},  
\begin{align}
\Phi_2^*(\xi) = \frac{1}{2}z(1-p) \xi^2 
- \left(1 + z\left(\frac{3}{2} - p \right)\right) \xi
+ 1 - \frac{1}{2}zp, 
\end{align}
and
\begin{align} 
\Phi_1(\xi) = \left(1 - zp + z^2 \left( \frac{1}{2} p - \frac{1}{4} \right) \right)\xi -1 - z(1-p) + z^2 \left( \frac{3}{4} - \frac{1}{2}p \right).
\end{align}
Next, we verify that if $p\geq 1$, then $\abs{\Phi_2(0)} < \abs{\Phi^*_2(0)}$. Reformulating the expression as
\begin{align*}
\abs{\Phi_2(0)} < \abs{\Phi^*_2(0)} 
\iff 
0 < \left(\Phi^*_2(0) \right)^2 - \Phi_2(0)^2
= 1 - zp - \frac{1}{2}z^2\left( \frac{1}{2} - p \right),
\end{align*}
(and keeping in mind that we ask this inequality to hold only for $z<0$,) we find that the contribution from each term  in the rightmost quadratic is positive, thus verifying the claim.
Finally, we show that $\Phi_1$ is simple von Neumann directly. Denoting the root of $\Phi_1$ as $\xi_1$, 
\begin{align*}
	\abs{\xi_1} < 1 
\iff \abs{\frac{(2p-3)z-2}{(2p-1)z-2}} < 1
\iff 0< 8z((p-1)z-1),  
\end{align*}
which holds for all $z<0$ if and only if $p\geq 1$.

Other IMEX schemes are analyzed in the same way. The amplification polynomials of the second order schemes are recorded for reference in Table \ref{table:amp poly 2}, along with the parameter restriction for which we observe unconditional stability.

\begin{table}[htb!]
	\centering
	\caption[Amplification polynomials of second order IMEX]{Amplification polynomial of second order IMEX schemes. The rightmost column is the guide for choosing $p$.}
	\begin{tabular}{lll}
		\toprule[1.25pt] 
		\head{Method} 
		& \head{Amplification Polynomial}
		& \head{$p\lambda/\lambda\in$}
		\\	\midrule 
		CNAB 
		& $\left(1 - \frac{1}{2}zp\right)\xi^2
		- \left(1 + z\left(\frac{3}{2}-p \right)\right)\xi + \frac{1}{2}z(1-p)$
		& $[1,\infty)$
		\\[2.6pt]
		mCNAB 
		& $\left(1 - \frac{9}{16}zp \right) \xi^2 - \left(1 + z\left(\frac{3}{2} - \frac{9}{8} p \right) \right)\xi
		+ \frac{1}{2}z\left(1 - \frac{9}{8}p \right) $
		& $[8/9,\infty)$
		\\[2.6pt]
		CNLF 
		& $\left(1-pz\right) \xi^2 -2z(1-p)\xi -(1+pz)$
		& $[1/2,\infty)$
		\\[2.6pt]
		SBDF2 
		& $\left(\frac{3}{2} - zp\right) \xi^2
		- 2\left(1 + z(1-p)\right) \xi 
		+ \frac{1}{2} + z(1-p)
		$
		& $[3/4,\infty)$
		\\ \bottomrule[1.25pt]
	\end{tabular}
\label{table:amp poly 2}
\end{table}

\subsection{Amplification polynomials of third and fourth order IMEX LMMs}
We continue with the analysis for higher order IMEX schemes. Again, amplification polynomials and parameter restrictions are derived. Because the expressions and manipulations quickly become cumbersome and tedious for higher order methods, the computer algebra system, \textsc{Maple}\texttrademark, was used to for the majority of the calculations.

\begin{table}[htb!]
	\centering
	\caption[Amplification polynomials of third and fourth order IMEX]{Amplification polynomial and choice of parameter when applying high order IMEX LMMs to the modified test equation for unconditional stability.}
	\begin{tabular}{lll}
		\toprule[1.25pt] 
		\head{Method} 
		& \head{Amplification Polynomial}
		& \head{$p\lambda/\lambda\in$}
		\\	\midrule 
		SBDF3
		& $\left(\frac{11}{6} - zp \right)\xi^3
		- 3\left(1 + z(1-p) \right) \xi^2 
		+ \frac{3}{2}\left(1 + 2z(1-p) \right) \xi 
		$
		& $[7/8, 2]$
		\\
		& \phantom{$\left(\frac{11}{6} - zp \right)\xi^3
			- 3\left(1 + z(1-p) \right) \xi^2$}$- \frac{1}{3}\left(1 +  3z(1-p) \right)$
		\\ [2.6pt]
		SBDF4
		& $\left(\frac{25}{12} - zp \right) \xi^4
		- 4\left(1 + z(1-p) \right)\xi^3
		+ 3\left(1 + 2z(1-p) \right)\xi^2 
		$
		& $[15/16, 5/4]$ 
		\\
		& \phantom{$\left(\frac{25}{12} - zp \right) \xi^4$}$- \frac{4}{3}\left(1 + 3z(1-p) \right) \xi
		+ \frac{1}{4}\left(1 + 4z(1-p)\right)$
		\\ \bottomrule[1.25pt]
	\end{tabular}
	\label{table:amp poly 34}
\end{table}

Listed in Table \ref{table:amp poly 34} are respectively the amplification factor and the parameter restriction for SBDF3 and SBDF4. We must point out a crucial difference. In contrast to the second order methods, the derived parameter restriction leaves only a finite interval.  This will be addressed further in the subsequent section where it is demonstrated that this detail renders the linearly stabilized SBDF3 and SBDF4 useless.

\section{Convergence of Linearly Stabilized IMEX Methods}
We present in this section numerical tests to support our claims. Stability contours are shown and a numerical convergence test to \cref{ammc} is conducted. We then provide an answer as to why linearly stabilized SBDF3 and SBDF4 fails. 

\subsection{Stability contours}
Presented in \cref{cnab contours,mcnab contours,cnlf contours,sbdf2 contours} are the stability contours for the second order IMEX schemes, \cref{cnab,mcnab,cnlf,sbdf2}, applied to the modified test equation, \cref{mte}. For each, we provide stability contours with $p$ set at $0.95p_0$, $p_0$, $1.5p_0$, where $p_0$ is the minimum required for unconditional stability (\cref{table:amp poly 2}, rightmost column). 

In \cref{sbdf3 contours} are the stability contours for SBDF3 \cref{sbdf3}, and in \cref{sbdf4 contours} are the stability contours for SBDF4 \cref{sbdf4}. In each, we plot four instances to capture finite range where unconditional stability is observed.
\begin{figure}[htb!]
        \centering
%\includegraphics[width=0.65\textwidth]{cnab_stab_contour}
\includegraphics[width=0.65\textwidth]{cnab_stab_contour2}
\caption{Stability contours for CNAB at various $p$.}
\label{cnab contours}
\end{figure}
\begin{figure}[htb!]
        \centering
%\includegraphics[width=0.65\textwidth]{mcnab_stab_contour}
\includegraphics[width=0.65\textwidth]{mcnab_stab_contour2}
\caption{Stability contours for mCNAB at various $p$.}
\label{mcnab contours}
\end{figure}
\begin{figure}[htb!]
        \centering
%\includegraphics[width=0.65\textwidth]{cnlf_stab_contour}
\includegraphics[width=0.65\textwidth]{cnlf_stab_contour2}
\caption{Stability contours for CNLF at various $p$.}
\label{cnlf contours}
\end{figure}
\begin{figure}[htb!]
        \centering
%\includegraphics[width=0.65\textwidth]{sbdf2_stab_contour}
\includegraphics[width=0.65\textwidth]{sbdf2_stab_contour2}
\caption{Stability contours for SBDF2 at various $p$.}
\label{sbdf2 contours}
\end{figure}

\begin{figure}[htb!]
        \centering
%\includegraphics[width=0.65\textwidth]{sbdf3_stab_contour}
\includegraphics[width=0.65\textwidth]{sbdf3_stab_contour2}
\caption{Stability contours for SBDF3 at various $p$.}
\label{sbdf3 contours}
\end{figure}

\begin{figure}[htb!]
        \centering
%\includegraphics[width=0.65\textwidth]{sbdf4_stab_contour}
\includegraphics[width=0.65\textwidth]{sbdf4_stab_contour2}
\caption{Stability contours for SBDF4 at various $p$.}
\label{sbdf4 contours}
\end{figure}

\subsection{Numerical convergence test}
Convergence of the proposed schemes will be tested on the 1D mean curvature motion problem, which we restate below:
\begin{subequations}
\begin{align}
        u_t = \frac{u_{xx}}{1 + u_x^2} - \frac{1}{u} - pu_{xx} + pu_{xx},
\quad 0< x< 10,\quad t>0,
\end{align}
with initial and boundary conditions
\begin{gather}
        u(x,0) = 1 + 0.10\sin\left(\frac{\pi}{5}x \right) 
\\
u(0,t) = u(10,t) = 1.
\end{gather}
\label{ammc pmpm}
\end{subequations}

As in \cref{ein conv test}, we solve to time $T=0.35$ with $N=2048$ spatial grid nodes. A reference solution is generated using a third order Runge-Kutta method with time step-size $\Delta t = \num{1.46e-5}$, and a max norm relative error calculated. Starting values necessary for the multistep methods are found using a third order Runge-Kutta method with small time step. The values of $p$ used for each scheme is as indicated in \cref{lmm conv test}.

Results of the numerical convergence study are shown in \cref{lmm conv test}. Each of the second order methods converge with the order of accuracy as expected, with SBDF2 having the largest errors, followed by CNAB/mCNAB (identical performance), CNLF, and then EIN. SBDF3 converges nicely with $p=0.875$. SBDF4 does not exhibit fourth order convergence and in fact fails at both $\Delta t =\num{6.84e-04}$ and $\Delta t=\num{3.42e-04}$. We discuss next why SBDF4 fails, and show also that SBDF3 suffers from the same defect.

\begin{figure}
        \centering
\includegraphics[width=0.65\textwidth]{lmm_conv_ammc}
\caption[Numerical convergence study with IMEX LMMs]{Numerical convergence study to \cref{ammc pmpm} with IMEX methods. Convergence of EIN is also included for comparison. Test with mCNAB is omitted, but would otherwise overlap with CNAB.}
\label{lmm conv test}
\end{figure}

\subsection{Failure with high order IMEX methods}
First, let us tabulate the observed (non)convergence of SBDF3 at various values of $p$. In \cref{table: sbdf3 nonconvergence}, we document three cases. The first case is the one shown in \cref{lmm conv test}. The second case exhibits a drastic drop in the observed convergence rate and and in the third case the method diverged as the time step-size is reduced. We attribute the divergence of SBDF3 and SBDF4 to the fact that their parameter restriction is a finite interval.

\begin{table}[htb!]
        \centering
\caption[(Non)convergence of SBDF3 at various $p$.]{(Non)convergence of SBDF3 at various $p$.}
\begin{tabular}{lccc} \toprule[1.25pt]
& \multicolumn{3}{c}{Observed convergence rate}
\\
        $\Delta t$ & $p=0.875$ & $p=1.475$ & $p=1.675$ 
\\ \midrule
\num{2.19e-2} & -- & -- & -- 
\\ 
\num{1.09e-2} & \num{2.39} & \num{2.39} & \num{2.39}
\\
\num{5.47e-3} & \num{2.64} & \num{2.64} & \num{1.23}
\\
\num{2.73e-3} & \num{2.80} & \num{2.80} & \num{-1.51}
\\ 
\num{1.37e-3} & \num{2.90} & \num{2.90} & \num{-3.95}
\\
\num{6.84e-4} & \num{2.95} & \num{2.95} & diverge
\\ 
\num{3.42e-4} & \num{2.97} & \num{2.08} & diverge
\\ \bottomrule[1.25pt]
\end{tabular}
\label{table: sbdf3 nonconvergence}
\end{table}

To see this, recall the parameter restrictions listed in \cref{table:amp poly 34} and the relation \cref{ammc eig}. Combining, we have for SBDF3 the parameter restriction
\begin{align}
        \frac{7}{8(1 + (D_1\bar u^n_j)^2)} \leq p \leq \frac{2}{1 + (D_1 \bar u^n_j)^2},
\quad j=1,\dots,N,
\end{align}
or equivalently
\begin{align} 
\max_{1\leq j\leq N}\frac{7}{8(1 + (D_1\bar u^n_j)^2)} \leq p 
\leq \min_{1\leq j\leq N}\frac{2}{1 + (D_1 \bar u^n_j)^2}.
\end{align}
Failure of the method is due to being unable to satisfy the parameter constraint at every grid node simultaneously. From \cref{fig:ammc sol}, we see that $\max_j (D_1 \bar u^n_j)^2$ is increasing as the solution evolves and is most extreme near the boundaries. Thus we expect instabilities to develop, and to develop in those regions first. This analysis is corroborated by \cref{fig:sbdf3 instab}, where we see instabilities developing near the righthand boundary.
\begin{figure}[htb!]
        \centering
\begin{minipage}{0.45\textwidth}
       \includegraphics[width=0.95\textwidth]{sbdf3_instabilities}
\end{minipage}
\begin{minipage}{0.45\textwidth}
       \includegraphics[width=0.95\textwidth]{sbdf3_instabilities_zoom}
\end{minipage}
\caption[Instabilities using SBDF3.]{Instabilities using SBDF3. With $p=1.675$ and $\Delta t=\num{9.2e-4}$, we observe instabilities developing near the righthand boundary of the gray dashed curve. The figure on the right is a zoom-in to the righthand boundary.}
\label{fig:sbdf3 instab}
\end{figure}

With SBDF4, the issue is only worse as the restriction is tighter. The results in \cref{lmm conv test} only appeared acceptable at coarse step-sizes because the low number of time steps did not allow for the instabilities to amplify and dominate the solution. We conclude that linear stabilization with SBDF3 and SBDF4  is not recommended. 

A natural follow-up is then to ask if all third and fourth order IMEX schemes are unsuited for combination with linear stabilization. To this we provide a partial answer in the negative. Third order, three step schemes form a three parameter family, and likewise for the fourth order, four step schemes \cite{ascher1995implicit}. An extensive search through the parameter space so far has yielded no positives, nor is there any evidence suggesting we may find one with unbounded $p$-parameter restriction.

This leaves us a number of competing second order methods. Our next task is to compare the performance of our IMEX based schemes vs.\ the EIN method of \cite{duchemin2014explicit}.

\section{Comparison of the Second Order Methods}
Of the methods that we have proposed, only the second order variants remain. Along with the EIN method, that gives five second order linearly stabilized schemes. A performance evaluation is in order.

\subsection{A 2D test problem}
Let us use as a test problem, the following nonlinear PDE from \cite{vdHouwen1982on}:
\begin{subequations}
        \begin{align}
                u_t = \Delta (u^5),
\quad 0\leq x,y\leq 1,
\quad t > 0,
\label{nl5a}
        \end{align}
with initial and boundary conditions set so that the exact solution is 
\begin{align}
u(x,y,t) = \left(\frac{4}{5}(2t+x+y)\right)^{1/4}.
\label{nl5b}
\end{align}
\label{nl5}%
\end{subequations}
Further assuming a uniform grid and centred differences in space, the eigenvalues of the Jacobian of the linearization are estimated to be in the interval
\begin{align}
        \left(-\frac{64}{h^2}(1+t), -16\pi^2(t+h) \right).
\label{nl5 eigs}
\end{align}

To \cref{nl5}, we propose to stabilize with a $p\Delta u$ term, i.e., replace \cref{nl5a} with 
\begin{align}
        u_t = \Delta (u^5) - p\Delta u + p\Delta u, \quad 0\leq x,y\leq 1,
\quad t > 0.
\label{nl5 p}
\end{align}
The parameter $p$ will then be chosen according to the ratio
\begin{align}
        \frac{p\lambda_\L}{\lambda_\N}
\approx \frac{-8p/h^2}{-64(1+t)/h^2}
= \frac{p}{8(1+t)}.
\end{align}
Let us remark on the importance of this test problem. In the analysis of \cref{ammc p}, we performed a von Neumann analysis to obtain precise eigenvalue estimates. On the contrary, the interval \cref{nl5 eigs} provides a bound that we do not expect to be sharp. We also never update $p$. It is chosen once and fixed at that value as we time step. Consequently, $p$ may be substantially greater than necessary. Moreover, to compensate for the fact that the operator we introduced to stabilize is less stiff than the nonlinear term, we are forced to select a relatively large value for $p$. We expect the situations just described to be common in applications and thus it is of interest to investigate performance of the methods and behaviour of its discretization errors.

\subsection{Loss of accuracy with EIN}
We test our second order methods on \cref{nl5 p} with initial and boundary conditions set by \cref{nl5b}. We solve to time $T=0.40$ with spatial grid size $\Delta x = \Delta y = h =0.015$. As a reference solution, we use an explicit third order Runge-Kutta method with time step-size $\Delta t=\num{6.25e-6}$. (Typical explicit methods demand step-sizes comparable for stability.) With the linearly stabilized schemes, we will solve \cref{nl5 p} with time step-sizes as large as $\Delta t = \num{1.25e-2}$. We show the max norm relative error.

\begin{figure}[htb!]
        \centering
\includegraphics[width=0.65\textwidth]{nl5_conv}
\caption[Numerical convergence study comparing second order methods.]{Numerical convergence study comparing second order methods. Test with mCNAB ($p=9.97$) is omitted, but would otherwise overlap with CNAB.}
\label{fig:nl5 conv}
\end{figure}

Results of the numerical convergence test is plotted in \cref{fig:nl5 conv} and it paints a grim picture for the EIN method and for CNLF. We will first discuss the horrific performance of the EIN method, after which we comment on the relative performance of the IMEX based schemes. 

Contrasting the performance of EIN in \cref{lmm conv test} and \cref{fig:nl5 conv}, we observe a drastic reduction in the order of accuracy. In the former, EIN converged with second order accuracy and recorded, amongst the second order method, the lowest error at any fixed $\Delta t$ (although an analysis of error vs.\ computing time would treat it less favorably). In our latest test problem, we do not (yet) observe second order convergence. Further testing shows that the EIN method only begins to demonstrate full second order rate of convergence as we dip below $\Delta t=\num{1e-5}$.

In fact, we argue that this behaviour can already be observed in the original paper by Duchemin and Eggers \cite{duchemin2014explicit}. In their experiments with Hele-Shaw interface flow and with the Kuramoto-Sivashinsky equation, they fail to accurately reproduce the reference figures taken from prior publications \cite{hou1994removing,kassam2005fourth}. In both cases, a large value of $p$ was necessary to obtain the desired stability. 

We offer an explanation. In the simple case of the modified test equation \cref{mte}, we can apply the EIN method and derive the amplification factor \cref{ein amp fac}. A series expansion at $z=0$ gives 
\begin{align}
        \xi_{EIN}
= 1 + z + \frac{z^2}{2} + \frac{1}{2}(p-p^2)z^3 + \O(p^3z^4).
\end{align}
This expression deviates from the exact solution, $\exp(z)$, in the $z^3$ term  and shows a quadratic dependence on $p$. That is to say, the leading error term may be large if $p$ is large and $\Delta t$ is not sufficiently small to control the error. That is exactly what is observed in \cref{fig:nl5 conv}.

On the other hand, the methods we have proposed all show only a linear dependence on $p$ under the same analysis, and do not suffer catastropically when $p$ is large.

\subsection{Amplification factors at infinity}
In the last section, we discovered a deficiency of the EIN method. The leading error term, which is $\O(\Delta t^3)$, has a coefficient that is quadratic in $p$, which we argued lead to a reduced observed order of accuracy. Thus it appears that an effective linearly stabilized time stepping scheme must have this error coefficient small with respect to $p$. 

That, however, does not explain the equally horrific performance of CNLF or the sharp dip in the observed convergence of CNAB  near $\Delta t = \num{1e-2}$ (see \cref{fig:nl5 conv}). To posit an explanation, we must think back to our discussion on stability and amplification factors. In that discussion, explicit schemes were said to be ill-suited to stiff problems because they are conditionally stable and this imposes a severe step-size restriction. Time steps violating this restriction will excite instabilities and energize high frequencies. A closer look at our schemes shows that we do exactly this. At each time step, there is an explicit calculation of a stiff nonlinear term and the other half of the ``adding zero'' term. Although we have guaranteed that our schemes are stable, the presence of slow decaying instabilities and persistent high frequency modes can drive up the error and force us to use smaller time steps to adequately damp and get the expected convergence order.

To explore this aspect, we consider the result of the method's amplification factor as $\lambda\Delta t \to -\infty$. For example, with the EIN method, we found the amplification factor in \cref{ein amp fac}. As $z\to -\infty$, we find 
\begin{align}
        \lim_{z\to-\infty} \abs{\xi_{EIN}} = \abs{\frac{p^2-3p+2}{p^2}}.
\end{align}
For the multistep schemes, we first find the limiting expression of the amplification polynomial, and then take the larger of the absolute value of the two roots. For instance, with CNAB, we start from \cref{amp_cnab} and compute
\begin{align}
        \lim_{z\to-\infty} \frac{\Phi_2}{z}
= -\frac{1}{2}p\xi^2-\left(\frac{3}{2}-p\right)\xi-\frac{1}{2}p+\frac{1}{2}.
\end{align}
The amplification factor at infinity for CNAB is then 
\begin{align}
\max\left\{ \frac{\abs{p-1+\sqrt{-2p+1}}}{p}, \frac{\abs{1-p+\sqrt{-2p+1}}}{p}\right\}
\end{align}

\cref{fig:damp fac at inf} shows the amplification factor at infinity for all of our second order schemes as well as SBDF1, and explains the behaviour of CNLF. It is known that CNLF is weakly damping at high frequencies and should not be used for diffusive problems \cite{ascher1995implicit}. This is equally true in the linear stabilization framework. Selecting coarse time steps with this scheme gives very poorly damping and it is necessary to take small time steps to combat this deficiency.

\begin{figure}[htb!]
	\centering 
\includegraphics[width=0.65\textwidth]{damp_at_inf}
\caption[Amplification factors at infinity.]{Amplification factors at infinity. The normalization along the horizontal axis is with respect to the lower limit of the parameter restriction of each scheme.}
\label{fig:damp fac at inf}
\end{figure}

Out of our schemes, SBDF2 provides the most damping. CNAB (and mCNAB) may suffer from large errors at the coarsest time steps, but remains an useful alternative to SBDF2 as it exhibits lower errors as $\Delta t\to 0$. The scheme mCNAB was derived in \cite{ascher1995implicit} as an alternative to CNAB with stronger high frequency damping, however, in our framework, they perform identically. Thus there is little reason to consider mCNAB over CNAB.

Finally, let us remark on the plotting range along the horizontal axis and how it pertains to the implementation of linearly stabilized schemes. For one, we have chosen to plot in a normalized parameter space. There are many reasons for this, but the most compelling is from the point of view of how we use these schemes. Implementation of linearly stabilized schemes start by choosing $p$ to satisfy the constraint 
\begin{align}
      \frac{p\lambda_\L}{\lambda_\N}  \geq p_0 
\iff  \frac{p}{p_0} \geq \frac{\lambda_\N}{\lambda_\L},
\end{align}
where $p_0$ is the lower limit of the parameter restriction that guarantees unconditional stability, and where $\lambda_\L, \lambda_\N$ are the largest absolute eigenvalues of linear and nonlinear operators. The quantity $\lambda_\N/\lambda_\L$ is independent of the choice of time stepping method and if this quantity were overestimated in our solution procedure, the effect on $p/p_0$ is equal irrespective of the scheme. We have also plotted over a large range of parameter values. The reason is again rooted in the implementation. In our derivations, we operate as if we can access the exact eigenvalues at every time step. This is impractical and impossible. In practice, the ratio of eigenvalues will be an overestimate, and the value of $p$ is fixed throughout the time evolution (or at least for a great number of time steps.) Therefore it necessary to chart the behaviour of the methods over a wide range of $p$.
