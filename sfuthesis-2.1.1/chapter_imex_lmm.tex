\chapter{IMEX Linear Multistep Methods}
For equations whose righthand side comprise of a stiff linear component and a nonstiff nonlinear part, a popular class of methods to apply are the implicit-explicit linear multistep methods (IMEX LMMs). The simplest of these is the semi-implicit Euler -- forward Euler to the nonlinear component and backward Euler to the linear, stiff component -- a scheme that we reviewed in Chapter 2. 

In this chapter, we investigate the use of IMEX LMMs in the context of linearly stabilized schemes. The added linear term is solved implicitly to provide greater stability, and the remaining terms, including a stiff nonlinear term, are solved explicitly.

\section{IMEX LMM Formulas}
In \cite{ascher1995implicit}, IMEX LMMs up to order four are investigated and a select number of schemes are singled out for their extensive use in the literature or for desired properties such as strong high frequency damping. As we are familiar with the first order variant, we begin by listing the second order methods of interest. These, and the higher order variants, will be presented as applied to the ODE 
\begin{align*}
u' = f + g, 
\end{align*}
where $f$ we identify as the nonlinear/nonstiff component and $g$ the stiff component. 
\\ \noindent
\textbf{Second order methods}\\
CNAB:
\begin{align}
\frac{u^{n+1}-u^n}{\Delta t} 
= \frac{3}{2} f^n - \frac{1}{2}f^{n-1} 
+ \frac{1}{2}(g^{n+1} + g^n), 
\label{cnab}
\end{align}\\
mCNAB:
\begin{align}
\frac{u^{n+1}-u^n}{\Delta t} 
= \frac{3}{2}f^n - \frac{1}{2} f^{n-1}
+ \frac{9}{16}g^{n+1} 
+ \frac{3}{8}g^n
+ \frac{1}{16}g^{n-1},
\label{mcnab}
\end{align}
\\
CNLF:
\begin{align}
\frac{u^{n+1}-u^{n-1}}{2\Delta t}
= f^n + \frac{1}{2}(g^{n+1} + g^{n-1}),
\label{cnlf}
\end{align} \\
SBDF2:
\begin{align}
\frac{3u^{n+1}-4u^n+u^{n-1}}{2\Delta t} 
= 2f^n - f^{n-1} + g^{n+1}.
\label{sbdf2}
\end{align}

\noindent
\textbf{Third order methods}\\
SBDF3:
\begin{align}
\frac{1}{\Delta t}\left(\frac{11}{6}u^{n+1} - 3u^n + \frac{3}{2}u^{n-1} - \frac{1}{3}u^{n-2} \right) 
= 3f^n - 3f^{n-1} + f^{n-2} + g^{n+1}.
\label{sbdf3}
\end{align}

\noindent
\textbf{Fourth order methods}\\
SBDF4:
\begin{align}
\frac{1}{\Delta t}\left(\frac{25}{12}u^{n+1} - 4u^n + 3u^{n-1} - \frac{4}{3}u^{n-2} + \frac{1}{4}u^{n-3} \right) 
= 4f^n - 6f^{n-1} + 4f^{n-2} -f^{n-3} + g^{n+1}.
\label{sbdf4}
\end{align}

In the next section, we will apply these IMEX schemes to the modified test equation to determine for each scheme the range of $p$ suitable for linear stabilization.

\section{Analysis of the Amplification Polynomials}
Let us restate here the modified test equation and the basic assumptions we make.
The modified test equation is 
\begin{align*}
        u' = (1-p)\lambda u + p\lambda u,
\quad\text{where}\quad \lambda < 0\text{, } p>0.
\label{mtee}
\end{align*}
The goal is, for each IMEX scheme, to identify a constraint on the parameter $p$ such that when satisfied, we may freely choose the time step-size without being subject to a stability restriction (eg. $\Delta t < C\Delta x^2$).

\subsection{Schur and von Neumann polynomials}
The resulting polynomials from applying an $n$th order IMEX LMM to the modified test equation will be a degree $n$ polynomial in the amplification factor, $\xi$. The tool of choice for analyzing these amplification polynomials is the theory of von Neumann polynomials \cite[Chapter 4]{strikwerda2004finite}. Below we give the relevant definitions and theorems.

\begin{definition}
	The polynomial $\phi$ is a Schur polynomial if all its roots, $r_q$, satisfy $\abs{r_q} < 1$.
\end{definition}
\begin{definition}
        The polynomial $\phi$ is a von Neumann polynomial if all its roots, $r_q$, satisfy $\abs{r_q}\leq 1$.
\end{definition}
\begin{definition}
	The polynomial $\phi$ is a simple von Neumann polynomial is $\phi$ is a von Neumann polynomial and its roots on the unit circle are simple roots.
\end{definition}
\begin{definition}
	For any polynomial $\phi(\xi) = \sum^n_{j=0} a_j\xi^j$, we define the polynomial $\phi^*$ by $\phi^*(\xi) = \sum^n_{j=0} a^*_{n-j} \xi^j$, where $^*$ on the coefficient denotes the complex conjugate.
\label{defn:conj}
\end{definition}
\begin{definition}
	For any polynomial $\phi_n(\xi) = \sum^n_{j=0} a_j\xi^j$, we define recursively the polynomial $\phi_{n-1}$ by
	\begin{align}
	\phi_{n-1}(\xi) = \frac{\phi_n^*(0)\phi_n(\xi) - \phi_n(0)\phi_n^*(\xi)}{\xi}.
	\end{align} 
\label{defn:recurse}
\end{definition}
\begin{theorem}
	$\phi_n$ is a simple von Neumann polynomial if and only if either 
	\begin{enumerate}[label=(\alph{*})]
		\item $\abs{\phi_n(0)} < \abs{\phi^*_n(0)}$ and $\phi_{n-1}$ is a simple von Neumann polynomial or
		
		\item $\phi_{n-1}$ is identically zero and $\phi'_{n}$ is a Schur polynomial.
	\end{enumerate}
\label{thm:simple vN}
\end{theorem}

\subsection{Amplification polynomials of second order IMEX LMMs}
We start by applying CNAB \eqref{cnab} to the modified test equation \eqref{mte}. Combined with the anzatz $u^n = \xi^n$ and setting $z=\lambda\Delta t$, we get the amplification polynomial
\begin{align}
\Phi_2(\xi) 
= \left(1 - \frac{1}{2}zp\right)\xi^2
- \left(1 + z\left(\frac{3}{2}-p \right)\right)\xi + \frac{1}{2}z(1-p).
\label{amp_cnab}
\end{align}
The next series of steps will show that for all $z<0$ (i.e. $\lambda < 0$,) \eqref{amp_cnab} is a simple von Neumann polynomial if and only if $p\geq 1$. We do so by showing that \cref{thm:simple vN} holds for $\Phi_2$.

We first give $\Phi_2^*$ and $\Phi_1$ as defined by the processes in \cref{defn:conj,defn:recurse},  
\begin{align}
\Phi_2^*(\xi) = \frac{1}{2}z(1-p) \xi^2 
- \left(1 + z\left(\frac{3}{2} - p \right)\right) \xi
+ 1 - \frac{1}{2}zp, 
\end{align}
and
\begin{align} 
\Phi_1(\xi) = \left(1 - zp + z^2 \left( \frac{1}{2} p - \frac{1}{4} \right) \right)\xi -1 - z(1-p) + z^2 \left( \frac{3}{4} - \frac{1}{2}p \right).
\end{align}
Next, we verify that if $p\geq 1$, then $\abs{\Phi_2(0)} < \abs{\Phi^*_2(0)}$. Reformulating the expression as
\begin{align*}
\abs{\Phi_2(0)} < \abs{\Phi^*_2(0)} 
\iff 
0 < \left(\Phi^*_2(0) \right)^2 - \Phi_2(0)^2
= 1 - zp - \frac{1}{2}z^2\left( \frac{1}{2} - p \right),
\end{align*}
(and keeping in mind that we ask this inequality to hold only for $z<0$,) we find that the contribution from each term  in the rightmost quadratic is positive, thus verifying the claim.
Finally, we show that $\Phi_1$ is simple von Neumann directly. Denoting the root of $\Phi_1$ as $\xi_1$, 
\begin{align*}
	\abs{\xi_1} < 1 
\iff \abs{\frac{(2p-3)z-2}{(2p-1)z-2}} < 1
\iff 0< 8z((p-1)z-1),  
\end{align*}
which holds for all $z<0$ if and only if $p\geq 1$.

Other IMEX LMMs are analyzed in the same way. Their amplification polynomials are recorded for reference in Table \ref{table:amp poly 2}, along with the parameter restriction for which we observe unconditional stability.

\begin{table}[htb!]
	\centering
	\caption[Amplification polynomials of second order IMEX]{Amplification polynomial of second order IMEX schemes. The rightmost column is the guide for choosing $p$.}
	\begin{tabular}{lll}
		\toprule[1.5pt] 
		\head{Method} 
		& \head{Amplification Polynomial}
		& \head{$p\lambda/\lambda\in$}
		\\	\midrule 
		CNAB 
		& $\left(1 - \frac{1}{2}zp\right)\xi^2
		- \left(1 + z\left(\frac{3}{2}-p \right)\right)\xi + \frac{1}{2}z(1-p)$
		& $[1,\infty)$
		\\[2.6pt]
		mCNAB 
		& $\left(1 - \frac{9}{16}zp \right) \xi^2 - \left(1 + z\left(\frac{3}{2} - \frac{9}{8} p \right) \right)\xi
		+ \frac{1}{2}z\left(1 - \frac{9}{8}p \right) $
		& $[8/9,\infty)$
		\\[2.6pt]
		CNLF 
		& $\left(1-pz\right) \xi^2 -2z(1-p)\xi -(1+pz)$
		& $[1/2,\infty)$
		\\[2.6pt]
		SBDF2 
		& $\left(\frac{3}{2} - zp\right) \xi^2
		- 2\left(1 + z(1-p)\right) \xi 
		+ \frac{1}{2} + z(1-p)
		$
		& $[3/4,\infty)$
		\\ \bottomrule[1.5pt]
	\end{tabular}
\label{table:amp poly 2}
\end{table}

\subsection{Amplification polynomials of third and fourth order IMEX LMMs}
We continue with the analysis for higher order IMEX LMMs. Again, amplification polynomials and parameter restrictions are derived. Because the expressions and manipulations quickly become cumbersome and tedious for higher order methods, the computer algebra system, \textsc{Maple}, was used to for the majority of the calculations.

\begin{table}[htb!]
	\centering
	\caption[Amplification polynomials of third and fourth order IMEX]{Amplification polynomial and choice of parameter when applying high order IMEX LMMs to the modified test equation for unconditional stability.}
	\begin{tabular}{lll}
		\toprule[1.5pt] 
		\head{Method} 
		& \head{Amplification Polynomial}
		& \head{$p\lambda/\lambda\in$}
		\\	\midrule 
		SBDF3
		& $\left(\frac{11}{6} - zp \right)\xi^3
		- 3\left(1 + z(1-p) \right) \xi^2 
		+ \frac{3}{2}\left(1 + 2z(1-p) \right) \xi 
		$
		& $[7/8, 2]$
		\\
		& \phantom{$\left(\frac{11}{6} - zp \right)\xi^3
			- 3\left(1 + z(1-p) \right) \xi^2$}$- \frac{1}{3}\left(1 +  3z(1-p) \right)$
		\\ [2.6pt]
		SBDF4
		& $\left(\frac{25}{12} - zp \right) \xi^4
		- 4\left(1 + z(1-p) \right)\xi^3
		+ 3\left(1 + 2z(1-p) \right)\xi^2 
		$
		& $[15/16, 5/4]$ 
		\\
		& \phantom{$\left(\frac{25}{12} - zp \right) \xi^4$}$- \frac{4}{3}\left(1 + 3z(1-p) \right) \xi
		+ \frac{1}{4}\left(1 + 4z(1-p)\right)$
		\\ \bottomrule[1.5pt]
	\end{tabular}
	\label{table:amp poly 34}
\end{table}

Listed in Table \ref{table:amp poly 34} are respectively the amplification factor and the parameter restriction for SBDF3 and SBDF4. We must point out a crucial difference. In contrast to the second order methods, the derived parameter restriction leaves only a finite interval.  This will be addressed further in the subsequent section where it is demonstrated that this detail renders the linearly stabilized SBDF3 and SBDF4 useless.

\section{Numerical Results}
We present in this section some numerical tests to support our claims. Stability contours and a numerical convergence test to \cref{ammc} are first shown. Then we provide an answer as to why SBDF3 and SBDF4 fails. Following that, we conduct a detailed comparison of the second order methods, including the EIN method (see \cref{sect:ein}).

\subsection{Stability contours}
Presented in \cref{cnab contours,mcnab contours,cnlf contours,sbdf2 contours} are the stability contours for the second order IMEX LMMs, \cref{cnab,mcnab,cnlf,sbdf2}, applied to the modified test equation, \cref{mte}. In \cref{sbdf3 contours} are the stability contours for SBDF3 \cref{sbdf3}, and \cref{sbdf4 contours} show the stability contours for SBDF4 \cref{sbdf4}. They are in agreement with the analysis summarized in \cref{table:amp poly 2,table:amp poly 34}
\begin{figure}[htb!]
        \centering
\includegraphics[width=0.65\textwidth]{cnab_stab_contour}
\caption{Stability contours for CNAB with various $p$.}
\label{cnab contours}
\end{figure}
\begin{figure}[htb!]
        \centering
\includegraphics[width=0.65\textwidth]{mcnab_stab_contour}
\caption{Stability contours for mCNAB with various $p$.}
\label{mcnab contours}
\end{figure}
\begin{figure}[htb!]
        \centering
\includegraphics[width=0.65\textwidth]{cnlf_stab_contour}
\caption{Stability contours for CNLF with various $p$.}
\label{cnlf contours}
\end{figure}
\begin{figure}[htb!]
        \centering
\includegraphics[width=0.65\textwidth]{sbdf2_stab_contour}
\caption{Stability contours for SBDF2 with various $p$.}
\label{sbdf2 contours}
\end{figure}

\begin{figure}[htb!]
        \centering
\includegraphics[width=0.65\textwidth]{sbdf3_stab_contour}
\caption{Stability contours for SBDF3 with various $p$.}
\label{sbdf3 contours}
\end{figure}

\begin{figure}[htb!]
        \centering
\includegraphics[width=0.65\textwidth]{sbdf4_stab_contour}
\caption{Stability contours for SBDF4 with various $p$.}
\label{sbdf4 contours}
\end{figure}

\subsection{Numerical convergence tests}
Convergence of the proposed schemes will be tested on three problems.

\subsubsection{1D mean curvature motion}
The first of the three is the 1D mean curvature motion problem \cref{ammc}, which we restate below:
\begin{subequations}
\begin{align}
        u_t = \frac{u_{xx}}{1 + u_x^2} - \frac{1}{u} - pu_{xx} + pu_{xx},
\quad 0< x< 10,\, t>0,
\end{align}
with initial and boundary conditions
\begin{gather}
        u(x,0) = 1 + 0.10\sin\left(\frac{\pi}{5}x \right) 
\\
u(0,t) = u(10,t) = 1.
\end{gather}
\label{ammc pmpm}
\end{subequations}

As in \cref{ein conv test}, we solve to time $T=0.35$ with $N=2048$ spatial grid nodes, and a reference solution is generated using a third order Runge-Kutta method with time step-size $\Delta t = \num{1.46e-5}$. Starting values necessary for the multistep methods are found using a third order Runge-Kutta method with small time step. The values of $p$ used for each scheme is indicated in \cref{lmm conv test}.

Results of the numerical convergence study are shown in \cref{lmm conv test}. Each of the second order methods converge with the order of accuracy as expected, with SBDF2 having the largest errors, followed by CNAB/mCNAB (identical performance), CNLF, and EIN. SBDF3 converges nicely for $p=0.875$. SBDF4 does not exhibit fourth order convergence and in fact fails at both $\Delta t =\num{6.84e-04}$ and $\Delta t=\num{3.42e-04}$. We discuss next why SBDF4 fails, and show also that SBDF3 suffers from the same defect.

\begin{figure}
        \centering
\includegraphics[width=0.65\textwidth]{lmm_conv_ammc}
\caption[Numerical convergence study with IMEX LMMs]{Numerical convergence study to \cref{ammc pmpm} with IMEX LMMs. Convergence of EIN is also included for comparison. Test with mCNAB is omitted, but would otherwise overlap with CNAB.}
\label{lmm conv test}
\end{figure}

\subsubsection{Failure with high order IMEX LMMs}
First, let us tabulate the observed (non)convergence of SBDF3 at various values of $p$. In \cref{table: sbdf3 nonconvergence} we document a case exhibiting reduction in the order of accuracy and a case where the method diverged as the time step-size was reduced. We attribute the divergence of SBDF3 and SBDF4 to the fact that their parameter restriction leaves only a finite interval.

\begin{table}[htb!]
        \centering
\caption[(Non)convergence of SBDF3 with various $p$.]{(Non)convergence of SBDF3 with various $p$.}
\begin{tabular}{lccc} \toprule[1.5pt]
& \multicolumn{3}{c}{Observed convergence rate}
\\
        $\Delta t$ & $p=0.875$ & $p=1.475$ & $p=1.675$ 
\\ \midrule
\num{2.19e-2} & -- & -- & -- 
\\ 
\num{1.09e-2} & \num{2.39} & \num{2.39} & \num{2.39}
\\
\num{5.47e-3} & \num{2.64} & \num{2.64} & \num{1.23}
\\
\num{2.73e-3} & \num{2.80} & \num{2.80} & \num{-1.51}
\\ 
\num{1.37e-3} & \num{2.90} & \num{2.90} & \num{-3.95}
\\
\num{6.84e-4} & \num{2.95} & \num{2.95} & diverge
\\ 
\num{3.42e-4} & \num{2.97} & \num{2.08} & diverge
\\ \bottomrule[1.5pt]
\end{tabular}
\label{table: sbdf3 nonconvergence}
\end{table}

To see this, recall the parameter restrictions listed in \cref{table:amp poly 34} and the relation \cref{ammc eig}. Combing we have for SBDF3 the parameter restriction
\begin{align}
        \frac{7}{8(1 + (D_1\bar u^n_j)^2)} \leq p \leq \frac{2}{1 + (D_1 \bar u^n_j)^2},
\quad j=1,\dots,N,
\end{align}
or equivalently
\begin{align} 
\max_{1\leq j\leq N}\frac{7}{8(1 + (D_1\bar u^n_j)^2)} \leq p 
\leq \min_{1\leq j\leq N}\frac{2}{1 + (D_1 \bar u^n_j)^2}.
\end{align}
Failure of the method is due to being unable to satisfy the parameter constraint at every grid node. From \cref{fig:ammc sol}, we see that $\max_j (D_1 \bar u^n_j)^2$ is increasing as the solution evolves and is most extreme near the boundaries. Thus we expect instabilities to develop, and to develop in those regions first. This analysis is corroborated by \cref{fig:sbdf3 instab}, where we see instabilities developing near the righthand boundary.
\begin{figure}[htb!]
        \centering
\includegraphics[width=0.65\textwidth]{sbdf3_instabilities}
\caption[Instabilities using SBDF3.]{Instabilities using SBDF3. With $p=1.675$ and $\Delta t=\num{9.2e-4}$, we observe instabilities developing near the righthand boundary.}
\label{fig:sbdf3 instab}
\end{figure}

With SBDF4, the issue is only worse as the restriction is tighter. The results in \cref{lmm conv test} for  SBDF4 only appeared acceptable at coarse step-sizes only because the low number of time steps did not allow for the instabilities to amplify and dominate the solution. We conclude that linear stabilization with SBDF3 and SBDF4  is not recommended. 

This raises the question: Are all third and fourth order IMEX LMMs unsuited for combination with linear stabilization? To this we provide a partial answer in the negative. Third order, three step schemes form a three parameter family, and likewise for the fourth order, four step schemes \cite{ascher1995implicit}. An extensive search through the parameter space so far has yielded no positives, nor is there any evidence suggesting we may find one with unbounded $p$-parameter restriction.

This leaves a number of competing second order methods. Our next task ...

\clearpage

+ cpu time comparison with EIN. What happens with the estimates for p are crude?

+ Further corroborate with Convergence test 2, nl5. Add CPU time comparison with EIN. cnlf fails. 

+ Further problems with EIN. Loss of accuracy.
