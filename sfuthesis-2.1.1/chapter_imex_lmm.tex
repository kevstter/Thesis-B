\chapter{IMEX Linear Multistep Methods}
For equations whose righthand side comprise of a stiff linear part and a nonstiff nonlinear component, a popular class of methods to apply are the implicit-explicit linear multistep methods (IMEX LMMs), with the most simple being an application of forward Euler to the nonlinear component and backward Euler to the linear, stiff component. 

\section{IMEX LMM Formulas}
In \cite{ascher1995implicit}, IMEX LMMs up to order four are investigated and a select number of schemes are singled out for their extensive use in the literature or for desired properties such as strong high frequency damping. As we are familiar with the first order variant, we begin by listing the second order methods of interest. These, and the higher order variants, will be presented as applied to the ODE 
\begin{align*}
u' = f + g, 
\end{align*}
where $f$ we identify as the nonlinear component and $g$ the stiff component. 
\\ \noindent
\textbf{Second order methods}\\
CNAB:
\begin{align}
\frac{u^{n+1}-u^n}{\Delta t} 
= \frac{3}{2} f^n - \frac{1}{2}f^{n-1} 
+ \frac{1}{2}(g^{n+1} + g^n), 
\label{cnab}
\end{align}\\
mCNAB:
\begin{align}
\frac{u^{n+1}-u^n}{\Delta t} 
= \frac{3}{2}f^n - \frac{1}{2} f^{n-1}
+ \frac{9}{16}g^{n+1} 
+ \frac{3}{8}g^n
+ \frac{1}{16}g^{n-1},
\label{mcnab}
\end{align}
\\
CNLF:
\begin{align}
\frac{u^{n+1}-u^{n-1}}{2\Delta t}
= f^n + \frac{1}{2}(g^{n+1} + g^{n-1}),
\label{cnlf}
\end{align} \\
SBDF2:
\begin{align}
\frac{3u^{n+1}-4u^n+u^{n-1}}{2\Delta t} 
= 2f^n - f^{n-1} + g^{n+1}.
\label{sbdf2}
\end{align}

\noindent
\textbf{Third order methods}\\
SBDF3:
\begin{align}
\frac{1}{\Delta t}\left(\frac{11}{6}u^{n+1} - 3u^n + \frac{3}{2}u^{n-1} - \frac{1}{3}u^{n-2} \right) 
= 3f^n - 3f^{n-1} + f^{n-2} + g^{n+1}.
\label{sbdf3}
\end{align}

\noindent
\textbf{Fourth order methods}\\
SBDF4:
\begin{align}
\frac{1}{\Delta t}\left(\frac{25}{12}u^{n+1} - 4u^n + 3u^{n-1} - \frac{4}{3}u^{n-2} + \frac{1}{4}u^{n-3} \right) 
= 4f^n - 6f^{n-1} + 4f^{n-2} -f^{n-3} + g^{n+1}.
\label{sbdf4}
\end{align}

Having given a collection of IMEX LMM formulas, we are now ready to apply such IMEX schemes to the modified test equation.

\section{Parameter Selection}
Let us restate here the modified test equation and some basic assumptions we make.
The modified test equation is 
\begin{align*}
        u' =  (1-p)\lambda u + p\lambda u.
\end{align*}
In our analysis, we will assume $p>0$ and $\lambda < 0$. The goal is then, for each IMEX LMM we consider, to identify a constraint on the parameter $p$, such that when satisfied, we may freely choose the time step-size without being subject to a stability restriction (eg. $\Delta t < C\Delta x^2$).

\subsection{Von Neumann Polynomials}
The resulting amplification polynomials from applying an $n$th order IMEX LMM to the modified test equation will be a degree $n$ polynomial in the amplification factor, $\xi$. The tool of choicie for the analysis of these amplification polynomials is the theory of Von Neumann polynomials \cite[Chapter 4]{strikwerda2004finite}. We begin with a definition.

\begin{definition}
        The polynomial $\phi$ is a von Neumann polynomial 
\end{definition}
