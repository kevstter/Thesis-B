\chapter{Adding Zero, Absolute Stability and a Modified Test Equation}
Our strategy for handling stiff, nonlinear problems involves, broadly speaking, making two choices. One is adding zero, and the other is choosing a time stepping method that provides the desired stability properties while handling the nonlinearity inexpensively. 

\section{Prototype 1D Problem}
As a prototype, let us consider the following 1D mean curvature motion problem \cite{duchemin2014explicit}: 
\begin{subequations} 
\begin{align}
u_t &= \frac{u_{xx}}{1 + u_x^2} - \frac{1}{u}, 
\qquad 0 < x < 10, \, t > 0,
\end{align}
with initial and boundary condition
\begin{gather}
u(x,0) = 1 + 0.10\sin\left( \frac{\pi}{5}x \right) 
\\
u(0,t)=u(10,t)=1.
\end{gather}
\label{ammc}
\end{subequations}
The presence of the $u_{xx}$ guarantees that \eqref{ammc} is stiff, suggesting that an implicit time stepping scheme would prove more efficient. However, to additionally handle the factor of $(1+u_x^2)^{-1}$ suggests otherwise. Thus we are presented with a scenario where neither an implicit nor an explicit approach proves particularly palatable.

\subsection{A first order unconditionally stable scheme}
As demonstrated in Duchemin and Eggers \cite{duchemin2014explicit} as well as an earlier paper by Smereka \cite{smereka2003semi}, an efficient method to handling this problem is to add and subtract a linear Laplacian term to the righthand side, 
\begin{align}
u_t = \underbrace{\frac{u_{xx}}{1 + u_x^2} 
- \frac{1}{u} 
- u_{xx}}_{\N(u)} 
+ \underbrace{\phantom{\frac{1_1}{1}}u_{xx}\phantom{\frac{1}{1_1}}}_{\L u}, 
\end{align} 
and then time step as 
\begin{align}
\frac{u^{n+1} - u^n}{\Delta t} 
= \N(u^n) + \L u^{n+1}.
\end{align}
Adopting a second order centred difference in space, we will show next that this scheme is unconditionally stable.

\subsection{A von Neumann stability analysis}
With the prescribed spatial-temporal discretization, we have at the interior nodes,
\begin{align}
\frac{u^{n+1}_j - u^n_j}{\Delta t}
= 4\frac{u^n_{j+1} - 2u^n_j + u^n_{j-1}}{4\Delta x^2 + (u^n_{j+1} - u^n_{j-1})^2}
- \frac{1}{u^n_j} 
- \frac{u^n_{j+1} - 2u^n_j + u^n_{j-1}}{\Delta x^2} 
+ \frac{u^{n+1}_{j+1} - 2u^{n+1}_j + u^{n+1}_{j-1}}{\Delta x^2}.
\label{fully_discrete_ammc}
\end{align} 
The von Neumann stability analysis then proceeds by writing the numerical solution as the exact solution to the difference equation \eqref{fully_discrete_ammc} perturbed by a single Fourier mode, 
\begin{align}
u^n_j = \bar u(j\Delta x, n\Delta t) 
+ \xi^n e^{ikj\Delta x}
= \bar u^n_j + \xi^n e^{ikj\Delta x}.
\end{align}
Recording the result term-by term, we have 
\begin{subequations}
	\begin{align}
	\frac{u^{n+1}_j - u^n_j}{\Delta t} 
&= \frac{\bar u^{n+1}_j - \bar u^n_j}{\Delta t} 
+ \frac{(\xi - 1)\xi^n e^{ikj\Delta t}}{\Delta t}, 
\\
 \frac{u^{n+1}_{j+1} - 2u^{n+1}_j + u^{n+1}_{j-1}}{\Delta x^2}
&= \frac{\bar u^{n+1}_{j+1} - 2\bar u^{n+1}_j + \bar u^{n+1}_{j-1}}{\Delta x^2}
+ \frac{2}{\Delta x^2}\left(\cos(k\Delta x) - 1\right) \xi^{n+1} e^{ikj\Delta x}
\\
	\frac{u^n_{j+1} - 2u^n_j + u^n_{j-1}}{\Delta x^2}
&= \frac{\bar u^n_{j+1} - 2\bar u^n_j + \bar u^n_{j-1}}{\Delta x^2} 
+ \frac{2}{\Delta x^2}\left(\cos(k\Delta x)-1\right) \xi^n e^{ikj\Delta x}
\\
	\frac{1}{u^n_j} 
&= \frac{1}{\bar u^n_j + \xi^n e^{ikj\Delta x}} 
\approx \frac{1}{\bar u^n_j} - \xi^n e^{ikj\Delta x} \frac{1}{(\bar u^n_j)^2}
	\end{align}
and 
\begin{align}
4\frac{u^n_{j+1} - 2u^n_j + u^n_{j-1}}{4\Delta x^2 + (u^n_{j+1} - u^n_{j-1})^2} 
&= 4\frac{\bar u^n_{j+1} - 2\bar u^n_j + \bar u^n_{j-1} + 2(\cos(k\Delta x)-1)\xi^ne^{ikj\Delta x}}{4\Delta x^2 + [\bar u^n_{j+1} - \bar u^n_{j-1} - 2i\sin(k\Delta x)\xi^ne^{ikj\Delta x}]^2}
\notag
\\
&= 4\frac{2(\cos(k\Delta x)-1)[\frac{\bar u^n_{j+1} - 2\bar u^n_j + \bar u^n_{j-1}}{2(\cos(k\Delta x)-1)} + \xi^n e^{ikj\Delta x}]}{4\Delta x^2 + (2i\sin(k\Delta x))^2[\frac{\bar u^n_{j+1} - \bar u^n_{j-1}}{2i\sin(k\Delta x)} - \xi^n e^{ikj\Delta x}]^2}
\notag\\
&\approx 4\frac{\bar u^n_{j+1} - 2\bar u^n_{j} + \bar u^n_{j-1}}{4\Delta x^2 + (\bar u^n_{j+1} - \bar u^n_{j-1})^2}
\notag\\
&+ 8(\cos(k\Delta x)-1)\xi^ne^{ikj\Delta x}\frac{1}{4\Delta x^2 + (\bar u^n_{j+1} - \bar u^n_{j-1})^2}
\notag\\ 
&-8i\sin(k\Delta x)\xi^ne^{ikj\Delta x}(\bar u^n_{j+1} - \bar u^n_{j-1})(\bar u^n_{j+1} - 2\bar u^n_{j} + \bar u^n_{j-1}) 
\notag\\
\begin{split} 
&\approx 4\frac{\bar u^n_{j+1} - 2\bar u^n_{j} + \bar u^n_{j-1}}{4\Delta x^2 + (\bar u^n_{j+1} - \bar u^n_{j-1})^2}
\\
&+ \frac{2}{\Delta x^2}(\cos(k\Delta x)-1)\xi^ne^{ikj\Delta x} \frac{1}{1 + (
	D_1 \bar u^n_j
	)^2},
\end{split}
\end{align}
\label{ammc_von_neumann}
\end{subequations}
where $D_1\bar u^n_j = (\bar u^n_{j+1} -  \bar u^n_{j-1})/(2\Delta x)$.
All together, \eqref{ammc_von_neumann} then simplifies to 
\begin{align}
        \frac{\xi - 1}{\Delta t} 
= \frac{2}{\Delta x^2}\frac{\cos(k\Delta x)-1}{1 +(
	D_1\bar u^n_j 
	)^2}
+ \frac{1}{(\bar u_j^n)^2}
+ \frac{2}{\Delta x^2}(\cos(k\Delta x)-1)(\xi-1),
\end{align}
from which we can then isolate the amplification factor,
\begin{align}
        \xi 
= 1+\underbrace{\Delta t \frac{\frac{2}{\Delta x^2}\frac{\cos(k\Delta x)-1}{1 +(
	D_1\bar u^n_j 
	)^2}
+ \frac{1}{(\bar u_j^n)^2}}{1-\frac{2\Delta t}{\Delta x^2}(\cos(k\Delta x)-1)}}_{=w}.
\end{align}
Absolute stability, $\abs{\xi} < 1$, is attained, if and only if $-2<w<0$. First, $w<0$. As the denominator is positive, this will hold so long as the numerator is negative, 
\begin{align}
        \frac{2}{\Delta x^2}\frac{\cos(k\Delta x)-1}{1 +(
	D_1\bar u^n_j 
	)^2}
+ \frac{1}{(\bar u_j^n)^2} < 0 
\iff 
\left(\frac{\Delta x}{\bar u^n_j} \right)^2 
< \frac{2(1-\cos(k\Delta x))}{1 + (D_1\bar u^n_j)^2}.
\end{align}
This last relation is satisfied on the assumption that $\Delta x \ll \bar u^n_j$.  Next, we examine the numerator of $w+2$ in order to show $w+2 > 0$,
\begin{align}
\begin{split}
  \frac{2\Delta t}{\Delta x^2}\frac{\cos(k\Delta x)-1}{1 + (D_1 \bar u^n_j)^2} + \frac{\Delta t}{(\bar u^n_j)^2} + 2 - 4\frac{\Delta t}{\Delta x^2}(\cos(k\Delta x)-1) 
\\
= 2\frac{\Delta t}{\Delta x^2}(\cos(k\Delta x)-1)\left(
\frac{1}{1 + (D_1 \bar u^n_j)^2} - 2 \right)
+ 2 + \frac{\Delta t}{(\bar u^n_j)^2}.
\end{split}
\end{align}
Each term is positive without restriction and thus $\abs{\xi} < 1$ for all $\Delta t > 0$.

\subsection{Second order by Richardson extrapolation}



















\newpage
 
In this section, it will be important to give exact details as to the type of equations that I am solving. Then starting from a generic equation of this type, we add zero
\begin{align}
        u_t = \N(u)
= \underbrace{p\L u}_\text{linear} +\underbrace{ (\N(u) - p\L u)}_\text{nonlinear}, 
\quad 
p > 0.
\label{generic}
\end{align}

We ask for stability for all $\Delta t > 0$. What is a parabolic PDE? Elliptic PDE? More on the history, the result of Duchemin and Eggers \cite{duchemin2014explicit}.


We can also set up an example to convey our idea. Use the problem from Duchemin and Eggers \cite{duchemin2014explicit}: 
\begin{align*}
u_t &= \frac{u_{xx}}{1 + u_x^2} - \frac{1}{u}, 
\qquad 0 < x < 10, \, t > 0,
\end{align*}
with initial and boundary condition
\begin{align*}
u(x,0) &= 1 + 0.10\sin\left(\frac{\pi}{5} x \right),
\\
u(0,t) &= u(10,t) = 1.
\end{align*}
We will this problem throughout this thesis as testing grounds to demonstrate our ideas.