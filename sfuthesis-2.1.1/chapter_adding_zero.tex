\chapter{Adding Zero, Absolute Stability and a Modified Test Equation}
Our strategy for handling stiff, nonlinear problems involves, broadly speaking, making two choices. One is adding zero, and the other is choosing a time stepping method that provides the desired stability properties while handling the nonlinearity inexpensively. 

\section{Prototype 1D Problem}
As a prototype, let us consider the following 1D mean curvature motion problem \cite{duchemin2014explicit}: 
\begin{subequations} 
\begin{align}
u_t &= \frac{u_{xx}}{1 + u_x^2} - \frac{1}{u}, 
\qquad 0 < x < 10, \, t > 0,
\end{align}
with initial and boundary condition
\begin{gather}
u(x,0) = 1 + 0.10\sin\left( \frac{\pi}{5}x \right) 
\\
u(0,t)=u(10,t)=1.
\end{gather}
\label{ammc}
\end{subequations}
The presence of the $u_{xx}$ guarantees that \eqref{ammc} is stiff, suggesting that an implicit time stepping scheme would prove more efficient. However, to additionally handle the factor of $(1+u_x^2)^{-1}$ suggests otherwise. Thus we are presented with a scenario where an implicit nor an explicit approach proves particularly palatable.

\subsection{A first order unconditionally stable scheme}
As demonstrated in Duchemin and Eggers \cite{duchemin2014explicit} as well as an earlier paper by Smereka \cite{smereka2003semi}, an efficient method to handling this problem is to add and subtract a linear Laplacian term to the righthand side, 
\begin{align}
u_t = \underbrace{\frac{u_{xx}}{1 + u_x^2} 
- \frac{1}{u} 
- u_{xx}}_{\N(u)} 
+ \underbrace{\phantom{\frac{1_1}{1}}u_{xx}\phantom{\frac{1}{1_1}}}_{\L u}, 
\end{align} 
and then time step as 
\begin{align}
\frac{u^{n+1} - u^n}{\Delta t} 
= \N(u^n) + \L u^{n+1}.
\end{align}
Adopting a second order centred difference in space, we will now show that this scheme is unconditionally stable.

With the prescribed spatial-temporal discretization, we have 
\begin{align}
\frac{u^{n+1}_j - u^n_j}{\Delta t}
= 4\frac{u^n_{j+1} - 2u^n_j + u^n_{j-1}}{4\Delta x^2 + (u^n_{j+1} - u^n_{j-1})^2}
- \frac{1}{u^n_j} 
- \frac{u^n_{j+1} - 2u^n_j + u^n_{j-1}}{\Delta x^2} 
+ \frac{u^{n+1}_{j+1} - 2u^{n+1}_j + u^{n+1}_{j-1}}{\Delta x^2}
\end{align} 

But the advantage of doing this is not immediate. Yes, the nonlinearity is handled explicitly and yes, there is now a stiff component handled implicitly, which perhaps will improve the stability. 

\newpage
 
In this section, it will be important to give exact details as to the type of equations that I am solving. Then starting from a generic equation of this type, we add zero
\begin{align}
        u_t = \N(u)
= \underbrace{p\L u}_\text{linear} +\underbrace{ (\N(u) - p\L u)}_\text{nonlinear}, 
\quad 
p > 0.
\label{generic}
\end{align}

We ask for stability for all $\Delta t > 0$. What is a parabolic PDE? Elliptic PDE? More on the history, the result of Duchemin and Eggers \cite{duchemin2014explicit}.


We can also set up an example to convey our idea. Use the problem from Duchemin and Eggers \cite{duchemin2014explicit}: 
\begin{align*}
u_t &= \frac{u_{xx}}{1 + u_x^2} - \frac{1}{u}, 
\qquad 0 < x < 10, \, t > 0,
\end{align*}
with initial and boundary condition
\begin{align*}
u(x,0) &= 1 + 0.10\sin\left(\frac{\pi}{5} x \right),
\\
u(0,t) &= u(10,t) = 1.
\end{align*}
We will this problem throughout this thesis as testing grounds to demonstrate our ideas.