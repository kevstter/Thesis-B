\chapter{Introduction}
In this thesis, we propose and analyze some new linearly stabilized schemes for the time integration of stiff nonlinear PDEs. A linearly stabilized scheme of first order has been used in a number of areas, with the first known example being from a paper by Douglas and Dupont \cite{douglas1971alternating} where they use this technique for the solution to a variable coefficient heat equation on rectangular domains. In subsequent years, the idea has been rediscovered by others \cite{eyre1998unconditionally,smereka2003semi} and has found applications to gradient systems, Hele-Shaw flow, interface motion, image processing, and solving PDEs on surfaces \cite{eyre1998bunconditionally,salac2008local,glasner2002diffuse,schonlieb2011unconditionally,macdonald2009implicit}.

In each of the references mentioned in the previous paragraph, the authors have succeeded in implemented only a first order time stepping method. Recently in \cite{duchemin2014explicit},  Duchemin and Eggers consolidated the approach and produced a second order linearly stabilized scheme they refer to as the explicit-implicit-null (EIN) method. The procedure they propose works off the first order scheme and extrapolates to second order. Moreover, they identified that the key principle for the success of any linearly stabilized scheme is unconditional stability. Indeed, a significant section of their paper is devoted to showing that their method is unconditionally stable under only a mild condition on a parameter that is introduced.

Our derivations for new linearly stabilized schemes also begins by ensuring that the newly derived schemes are in fact unconditionally stable. The techniques we employ in our stability analysis are very much those of a standard linear stability analysis and are reviewed first. This is then followed by Chapter 2 where we formally introduce the notion of linear stabilization. Movitation for this technique is supplied by the need to handled a 1D stiff nonlinear PDE describing axisymmetric mean curvature flow and leads us to the well-known first order linearly stabilized time integrator and the EIN method of Duchemin and Eggers. The framework in which we analyze the stability of linearly stabilized schemes is setup following. 

In Chapter 3 we investigate the use of implicit-explicit (IMEX) linear multistep method. A detailed comparison of the schemes based on IMEX multistep schemes and the EIN method is conducted. Our experiments suggest additional criteria for practical linearly stabilized schemes. 



\section{Linear Stability Analysis}



With increasing frequency, we (we? who?) find ourselves having to adopt nonlinear terms to correctly resolve complicated behaviours in interesting phenomena. Say things about interesting phenomena...

Our main interest is when the nonlinearity present is at the same time stiff. This causes considerable difficulties. 

Linearization ...

Stabilized explicit RK (RKC, ROCK, DUMKA)

Linear stabilization and prior work in applications and Smereka and Duch \& Eggers

Prior contributions to this subject: Duchemin and Eggers \cite{duchemin2014explicit}, Smereka \cite{smereka2003semi}. Other methods for handling PDEs of this type.
What are other ways people handle nonlinear stiff equations? 

Often, we see new, efficient solvers that are so complex, that the preferred method of choice remains those that are simple, clear, and easy to implement. 

\section{Overview}
An outline of this thesis is as follows. In Chapter 2, we discuss the notion of linear stabilization. The idea is not new and we will discuss some work that is relevant to our own. Stability is the central theme and an appropriate framework is set. 

After that, it's onto our work. We propose linear stabilization with IMEX linear multistep methods. We cover stability and implementation. Then we compare with a competing method and demonstrate our superiority in each and every way possible.

The framework supplied earlier is not limited to IMEX methods. We develop the same ideas with exponential Runge-Kutta methods for a higher order implementation. Following that we will compared with the IMEX class of linearly stabilized schemes.

Finally, we present a number of applications. Expect cool figures, some convergence plots, and some words.

To conclude, recap, future work, etc.