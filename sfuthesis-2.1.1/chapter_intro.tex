\chapter{Introduction}
With increasing frequency, we (we? who?) find ourselves having to adopt nonlinear terms to correctly resolve complicated behaviours in interesting phenomena. Say things about interesting phenomena...

Our main interest is when the nonlinearity present is at the same time stiff. This causes considerable difficulties. 

Linearization ...

Stabilized explicit RK (RKC, ROCK, DUMKA)

Linear stabilization and prior work in applications and Smereka and Duch \& Eggers

Prior contributions to this subject: Duchemin and Eggers \cite{duchemin2014explicit}, Smereka \cite{smereka2003semi}. Other methods for handling PDEs of this type.
What are other ways people handle nonlinear stiff equations? 

Often, we see new, efficient solvers that are so complex, that the preferred method of choice remains those that are simple, clear, and easy to implement. 

\section{Overview}
An outline of this thesis is as follows. In Chapter 2, we discuss the notion of linear stabilization. The idea is not new and we will discuss some work that is relevant to our own. Stability is the central theme and an appropriate framework is set. 

After that, it's onto our work. We propose linear stabilization with IMEX linear multistep methods. We cover stability and implementation. Then we compare with a competing method and demonstrate our superiority in each and every way possible.

The framework supplied earlier is not limited to IMEX methods. We develop the same ideas with exponential Runge-Kutta methods for a higher order implementation. Following that we will compared with the IMEX class of linearly stabilized schemes.

Finally, we present a number of applications. Expect cool figures, some convergence plots, and some words.

To conclude, recap, future work, etc.