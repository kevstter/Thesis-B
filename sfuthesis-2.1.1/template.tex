\documentclass{sfuthesis}

\title{Numerical Solution to Stiff PDEs}
\thesistype{Dissertation}
\author{Kevin Chow}
\previousdegrees{%
%	M.Sc., Wossamotta University, 1963\\
	B.Math., University of Waterloo, 2014}
\degree{Master of Science}
\discipline{Mathematics}
\department{Department of Mathematics}
\faculty{Faculty of Science}
\copyrightyear{2016}
\semester{Fall 2016}
\date{\today}
%\date{1 September 2015}

\keywords{Thesis template; Simon Fraser University; time travel paradoxes}

\committee{%
	\chair{Pamela Isely}{Professor}
	\member{Dr. Steven Ruuth}{Senior Supervisor\\Professor}
	\member{Dr. Robert Russell}{Professor Emerius}
	\member{James Moriarty}{Supervisor\\Adjunct Professor}
%	\member{Kaylee Frye}{Internal Examiner\\Assistant Professor\\School of Engineering Science}
%	\member{Hubert J.\ Farnsworth}{External Examiner\\Professor\\Department of Quantum Fields\\Mars University}
}

%   PACKAGES AND CUSTOMIZATIONS  %%%%%%%%%%%%%%%%%%%%%%%%%%%%%%%%%%%%%%%%%%%%%%
%
%   Add any packages or custom commands you need for your thesis here.
%   You don't need to call the following packages, which are already called in
%   the sfuthesis class file:
%
%   - appendix
%   - etoolbox
%   - fontenc
%   - geometry
%   - lmodern
%   - nowidow
%   - setspace
%   - tocloft
%
%   If you call one of the above packages (or one of their dependencies) with
%   options, you may get a "Option clash" LaTeX error. If you get this error,
%   you can fix it by removing your copy of \usepackage and passing the options
%   you need by adding
%
%       \PassOptionsToPackage{<options>}{<package>}
%
%   before \documentclass{sfuthesis}.
%

\usepackage{amsmath,amssymb,amsthm}
\usepackage[pdfborder={0 0 0}]{hyperref}
\usepackage[draft,demo]{graphicx}
%\usepackage{graphicx}
\usepackage{caption}
\usepackage{enumitem}
\usepackage{booktabs}
\usepackage[capitalize]{cleveref}
	\crefname{equation}{}{}
\usepackage{siunitx}
	%\sisetup{output-exponent-marker=\ensuremath{\mathrm{e}}}
\usepackage{natbib}
	\bibliographystyle{plain} %plain, abbrev, acm

\theoremstyle{plain}
\newtheorem{theorem}{Theorem}[chapter] 

\theoremstyle{definition}
\newtheorem{definition}[theorem]{Definition}
\newtheorem{example}[theorem]{Example}

\theoremstyle{remark}
\newtheorem{remark}[]{Remark}

\renewcommand{\L}{\mathcal{L}}
\newcommand{\N}{\mathcal{N}}
\renewcommand{\O}{\mathcal{O}}

\delimitershortfall-1sp
\newcommand{\abs}[1]{\left|#1\right|}

\newcommand{\head}[1]{\textnormal{\textbf{#1}}}

\newcommand{\dd}[2]{\frac{d#1}{d#2}}

\newcommand{\qand}{\quad\text{and}\quad}


%   FRONTMATTER  %%%%%%%%%%%%%%%%%%%%%%%%%%%%%%%%%%%%%%%%%%%%%%%%%%%%%%%%%%%%%%
%
%   Title page, committee page, copyright declaration, abstract,
%   dedication, acknowledgements, table of contents, etc.
%

\begin{document}

\frontmatter
\maketitle{}
\makecommittee{}

\begin{abstract}
	This is a blank document from which you can start writing your thesis.
\end{abstract}


\begin{dedication} % optional
\end{dedication}


\begin{acknowledgements} % optional
\end{acknowledgements}

\addtoToC{Table of Contents}\tableofcontents\clearpage
\addtoToC{List of Tables}\listoftables\clearpage
\addtoToC{List of Figures}\listoffigures





%   MAIN MATTER  %%%%%%%%%%%%%%%%%%%%%%%%%%%%%%%%%%%%%%%%%%%%%%%%%%%%%%%%%%%%%%
%
%   Start writing your thesis --- or start \include ing chapters --- here.
%

\mainmatter%

\setkeys{Gin}{draft}
\chapter{Introduction}
In this thesis, we propose and analyze some new linearly stabilized schemes for the time integration of stiff nonlinear PDEs. A linearly stabilized scheme of first order has been used in a number of areas, with the first known example being from a paper by Douglas and Dupont \cite{douglas1971alternating} where they use this technique for the solution to a variable coefficient heat equation on rectangular domains. In subsequent years, the idea has been rediscovered by others \cite{eyre1998unconditionally,smereka2003semi} and has found applications to gradient systems, Hele-Shaw flow, interface motion, image processing, and solving PDEs on surfaces \cite{eyre1998bunconditionally,salac2008local,glasner2002diffuse,schonlieb2011unconditionally,macdonald2009implicit}.

In each of the references mentioned in the previous paragraph, the authors have succeeded in implemented only a first order time stepping method. Recently in \cite{duchemin2014explicit},  Duchemin and Eggers consolidated the approach and produced a second order linearly stabilized scheme they refer to as the explicit-implicit-null (EIN) method. The procedure they propose works off the first order scheme and extrapolates to second order. Moreover, they identified that the key principle for the success of any linearly stabilized scheme is unconditional stability. Indeed, a significant section of their paper is devoted to showing that their method is unconditionally stable under only a mild condition on a parameter that is introduced.

Our derivations for new linearly stabilized schemes also begins by ensuring that the newly derived schemes are in fact unconditionally stable. The techniques we employ in our stability analysis are very much those of a standard linear stability analysis and are reviewed first. This is then followed by Chapter 2 where we formally introduce the notion of linear stabilization. Movitation for this technique is supplied by the need to handled a 1D stiff nonlinear PDE describing axisymmetric mean curvature flow and leads us to the well-known first order linearly stabilized time integrator and the EIN method of Duchemin and Eggers. Following that, the framework in which we analyze the stability of linearly stabilized schemes is set. 

In Chapter 3 we investigate implicit-explicit (IMEX) linear multistep methods within the linear stabilization framework. A detailed comparison of the schemes based on IMEX multistep schemes and the EIN method is conducted. Our experiments suggest criteria in addition to unconditional stability are necessary for practical linearly stabilized schemes. This in turn eliminates third order and higher multistep based linearly stabilized schemes from use. 

In Chapter 4, we explore the use of exponential Runge-Kutta methods to mend this deficiency. A second order and a fourth order exponential Runge-Kutta method are verified to exhibit parameter restrictions of the right form. However, other complications limits their use.

In Chapter 5, application of our linearly stabilized schemes to a number of 2D and 3D problems are presented. Not surprisingly, our second order schemes offer massive improvements over the commonly used first order linearly stabilized scheme. The experiments show that our schemes provide substantial efficiency improvement yet can be implemented with tremendous ease. 

Finally, some concluding remarks are presented in Chapter 6.  


\section{Linear Stability Analysis}
\subsection{Stability and the scalar test equation}
Linear stability analysis is predicated finding the constraints necessary on the time step-size so that the numerical solution generated using that particular time stepping method applied to the test equation, 
\begin{align}
u' = \lambda u, 
\quad \lambda < 0,
\label{test eqn}
\end{align}  
has properties resembling that of the exact solution to the test equation,
\begin{align}
u(t^n + \Delta t) = e^{\lambda \Delta t} u(t^n).
\end{align}
Observe that, in general, the exact solution satisfies
\begin{align}
\frac{\abs{u(t^n + \Delta t)}}{\abs{u(t^n)}}
= \abs{e^{\lambda\Delta t} } 
< 1,
\quad\text{for all } \Delta t > 0.
\end{align}
The analogous property for numerical methods is what we will refer to as stability.

For example, applying the forward Euler method to the test equation \cref{test eqn}, we get 
\begin{align}
\frac{u^{n+1} - u^n}{\Delta t} = \lambda u^n 
\iff u^{n+1} = \underbrace{(1 + \lambda \Delta t)}_{=\xi_{FE}} u^n, 
\end{align}
where $u^n$ is an approximation to $u(t^n)$. Then imposing $\abs{\xi_{FE}} < 1$, we get 
\begin{align}
\abs{1+ \lambda\Delta t} < 1 
\iff -2 < \lambda\Delta t < 0,
\end{align}
and so for stability, $\Delta t < -2/\lambda$ must be satisfied. As the time step-size, $\Delta t$, is constrained, we say forward Euler is conditionally stable. 

For another example, we may apply backward Euler to the test equation. Doing so we get 
\begin{align}
\frac{u^{n+1} - u^n}{\Delta t} = \lambda u^{n+1} 
\iff 
u^{n+1} = \underbrace{\frac{1}{1 - \lambda\Delta t}}_{=\xi_{BE}} u^n.
\end{align}
This time, imposing $\abs{\xi_{BE}}<1$ adds no new constraint to the time step-size. When no additional constraints are imposed on the time step-size, we say the numerical method is unconditionally stable.

More generally, to determine the stability constraint of any one step method, one applies said method to the test equation, rearranges as $u^{n+1} = \xi(\lambda\Delta t) u^n$, and imposes $\abs{\xi(\lambda\Delta t)} < 1$. The quantity $\xi(\lambda\Delta t)$ is commonly referred to as the amplification factor and the region where $\abs{\xi(\lambda\Delta t)} < 1$ the stability region (of the numerical method.) In other words, for stability, we require that the magnitude of the amplification factor is less than one.
 
\subsection{Stability contours}
A stability contour plot is a graphical device for understanding the stability constraint of a method. It offers a way for us to verify calculations done analytically, or to visualize the stability region of a numerical method where an analytic solution is infeasible. Stability contours plots in this thesis all show contours of the amplification factor $\xi(\lambda\Delta t)$ plotted over a subset of the region $\{\lambda\Delta t \in \bC \mid \Delta t > 0\}$, with a focus on the left half plane and in particular the negative real line. \cref{fig:FE BE stab cont} shows stability contours of the forward Euler and the backward Euler method. Note that the regions are symmetric with respect to the real axis and thus only $\Im(\lambda\Delta t) \geq 0$ will be plotted.

\begin{figure}[htb!]
	\centering
\begin{minipage}{0.45\textwidth}
\includegraphics[width=0.95\textwidth]{fe_stab}
\end{minipage}
\begin{minipage}{0.45\textwidth}
\includegraphics[width=0.95\textwidth]{be_stab}
\end{minipage}
\caption[Examples of stability contour plots.]{On the left is the stability contour plot for forward Euler. The stability region is the interior of the 1-contour. On the right is the stability contour plot for backward Euler. The stabilty region is the region outside the 1-contour.}
\label{fig:FE BE stab cont}
\end{figure}

\subsection{Relation to stiff PDEs}
Recall that our motivation is to develop methods suited to the time integration of stiff nonlinear PDEs. So how does the time step restriction of a numerical method derived from application to the test equation relate to time step selection for a stiff nonlinear PDE? The relation is as follows. Suppose the PDE has been discretized in space and we are to advance the solution of the resulting large system of ODEs, $u' = \N(u)$, by one time step, i.e.\ advance the numerical solution $u^n$ to $u^{n+1}$. Over just one time step, it may be reasonable to consider the linearization, 
\begin{align*}
        u' = \bar u + \pd{\N}{u} \bigg|_{u=\bar u} (u-\bar u), 
\end{align*}
or setting $v=u-\bar u$, $v' = Av$. Further assuming that $A$ is diagonalizable, $A=T^{-1}DT$, where $D=\diag(\lambda_1,\dots,\lambda_N)$, we get 
\begin{align}
        v' = T^{-1}DTv 
&\iff (Tv)' = D(Tv)
\\&\iff w_k' = \lambda w_k, \quad k=1,\dots, N.
\end{align}
In other words, under appropriate conditions, it may be fair to analyze the dynamics of the nonlinear system by inspecting the eigenvalues of the Jacobian from the linearized state. Thus to time step, we require that the computation be stable for each eigenmode. The time step constraint will then be dictated by the largest absolute eigenvalue.

For instance, suppose we found, from a linearized system of ODEs, the eigenvalues to be $2(\cos(k\Delta x) -1)/\Delta x^2$, $\Delta x = 1/N$. Then the largest absolute eigenvalue can be bounded as 
\begin{align}
        \abs{\frac{2}{\Delta x^2}(\cos(k\Delta x) -1)} \leq \frac{4}{\Delta x^2},
\end{align}
and stable time step-sizes for forward Euler must then satisfy $\Delta t < \Delta x^2/4$. On the other hand, unconditionally stable methods such as backward Euler, maintain stability irrespective of the grid size $\Delta x$.


\section{Overview}

With increasing frequency, we (we? who?) find ourselves having to adopt nonlinear terms to correctly resolve complicated behaviours in interesting phenomena. Say things about interesting phenomena...

Our main interest is when the nonlinearity present is at the same time stiff. This causes considerable difficulties. 

Linearization ...

Stabilized explicit RK (RKC, ROCK, DUMKA)

Linear stabilization and prior work in applications and Smereka and Duch \& Eggers

Prior contributions to this subject: Duchemin and Eggers \cite{duchemin2014explicit}, Smereka \cite{smereka2003semi}. Other methods for handling PDEs of this type.
What are other ways people handle nonlinear stiff equations? 

Often, we see new, efficient solvers that are so complex, that the preferred method of choice remains those that are simple, clear, and easy to implement. 


\chapter{Adding Zero, Absolute Stability and a Modified Test Equation}
Our strategy for handling stiff, nonlinear problems involves, broadly speaking, making two choices. One is adding zero, and the other is choosing a time stepping method that provides the desired stability properties while handling the nonlinearity inexpensively. 

\section{Prototype 1D Problem}
As a prototype, let us consider the following 1D mean curvature motion problem \cite{duchemin2014explicit}: 
\begin{subequations} 
\begin{align}
u_t &= \frac{u_{xx}}{1 + u_x^2} - \frac{1}{u}, 
\qquad 0 < x < 10, \, t > 0,
\end{align}
with initial and boundary condition
\begin{gather}
u(x,0) = 1 + 0.10\sin\left( \frac{\pi}{5}x \right) 
\\
u(0,t)=u(10,t)=1.
\end{gather}
\label{ammc}
\end{subequations}
The presence of the $u_{xx}$ guarantees that \eqref{ammc} is stiff, suggesting that an implicit time stepping scheme would prove more efficient. However, to additionally handle the factor of $(1+u_x^2)^{-1}$ suggests otherwise. Thus we are presented with a scenario where neither an implicit nor an explicit approach proves particularly palatable.

\subsection{A first order unconditionally stable scheme}
As demonstrated in Duchemin and Eggers \cite{duchemin2014explicit} as well as an earlier paper by Smereka \cite{smereka2003semi}, an efficient method to handling this problem is to add and subtract a linear Laplacian term to the righthand side, 
\begin{align}
u_t = \underbrace{\frac{u_{xx}}{1 + u_x^2} 
- \frac{1}{u} 
- u_{xx}}_{\N(u)} 
+ \underbrace{\phantom{\frac{1_1}{1}}u_{xx}\phantom{\frac{1}{1_1}}}_{\L u}, 
\end{align} 
and then time step as 
\begin{align}
\frac{u^{n+1} - u^n}{\Delta t} 
= \N(u^n) + \L u^{n+1}.
\end{align}
Adopting a second order centred difference in space, we will show next that this scheme is unconditionally stable.

\subsection{A von Neumann stability analysis}
With the prescribed spatial-temporal discretization, we have at the interior nodes,
\begin{align}
\frac{u^{n+1}_j - u^n_j}{\Delta t}
= 4\frac{u^n_{j+1} - 2u^n_j + u^n_{j-1}}{4\Delta x^2 + (u^n_{j+1} - u^n_{j-1})^2}
- \frac{1}{u^n_j} 
- \frac{u^n_{j+1} - 2u^n_j + u^n_{j-1}}{\Delta x^2} 
+ \frac{u^{n+1}_{j+1} - 2u^{n+1}_j + u^{n+1}_{j-1}}{\Delta x^2}.
\label{fully_discrete_ammc}
\end{align} 
The von Neumann stability analysis then proceeds by writing the numerical solution as the exact solution to the difference equation \eqref{fully_discrete_ammc} perturbed by a single Fourier mode, 
\begin{align}
u^n_j = \bar u(j\Delta x, n\Delta t) 
+ \xi^n e^{ikj\Delta x}
= \bar u^n_j + \xi^n e^{ikj\Delta x}.
\end{align}
Recording the result term-by term, we have 
\begin{subequations}
	\begin{align}
	\frac{u^{n+1}_j - u^n_j}{\Delta t} 
&= \frac{\bar u^{n+1}_j - \bar u^n_j}{\Delta t} 
+ \frac{(\xi - 1)\xi^n e^{ikj\Delta t}}{\Delta t}, 
\\
 \frac{u^{n+1}_{j+1} - 2u^{n+1}_j + u^{n+1}_{j-1}}{\Delta x^2}
&= \frac{\bar u^{n+1}_{j+1} - 2\bar u^{n+1}_j + \bar u^{n+1}_{j-1}}{\Delta x^2}
+ \frac{2}{\Delta x^2}\left(\cos(k\Delta x) - 1\right) \xi^{n+1} e^{ikj\Delta x}
\\
	\frac{u^n_{j+1} - 2u^n_j + u^n_{j-1}}{\Delta x^2}
&= \frac{\bar u^n_{j+1} - 2\bar u^n_j + \bar u^n_{j-1}}{\Delta x^2} 
+ \frac{2}{\Delta x^2}\left(\cos(k\Delta x)-1\right) \xi^n e^{ikj\Delta x}
\\
	\frac{1}{u^n_j} 
&= \frac{1}{\bar u^n_j + \xi^n e^{ikj\Delta x}} 
\approx \frac{1}{\bar u^n_j} - \xi^n e^{ikj\Delta x} \frac{1}{(\bar u^n_j)^2}
	\end{align}
and 
\begin{align}
4\frac{u^n_{j+1} - 2u^n_j + u^n_{j-1}}{4\Delta x^2 + (u^n_{j+1} - u^n_{j-1})^2} 
&= 4\frac{\bar u^n_{j+1} - 2\bar u^n_j + \bar u^n_{j-1} + 2(\cos(k\Delta x)-1)\xi^ne^{ikj\Delta x}}{4\Delta x^2 + [\bar u^n_{j+1} - \bar u^n_{j-1} - 2i\sin(k\Delta x)\xi^ne^{ikj\Delta x}]^2}
\notag
\\
&= 4\frac{2(\cos(k\Delta x)-1)[\frac{\bar u^n_{j+1} - 2\bar u^n_j + \bar u^n_{j-1}}{2(\cos(k\Delta x)-1)} + \xi^n e^{ikj\Delta x}]}{4\Delta x^2 + (2i\sin(k\Delta x))^2[\frac{\bar u^n_{j+1} - \bar u^n_{j-1}}{2i\sin(k\Delta x)} - \xi^n e^{ikj\Delta x}]^2}
\notag\\
&\approx 4\frac{\bar u^n_{j+1} - 2\bar u^n_{j} + \bar u^n_{j-1}}{4\Delta x^2 + (\bar u^n_{j+1} - \bar u^n_{j-1})^2}
\notag\\
&+ 8(\cos(k\Delta x)-1)\xi^ne^{ikj\Delta x}\frac{1}{4\Delta x^2 + (\bar u^n_{j+1} - \bar u^n_{j-1})^2}
\notag\\ 
&-8i\sin(k\Delta x)\xi^ne^{ikj\Delta x}(\bar u^n_{j+1} - \bar u^n_{j-1})(\bar u^n_{j+1} - 2\bar u^n_{j} + \bar u^n_{j-1}) 
\notag\\
\begin{split} 
&\approx 4\frac{\bar u^n_{j+1} - 2\bar u^n_{j} + \bar u^n_{j-1}}{4\Delta x^2 + (\bar u^n_{j+1} - \bar u^n_{j-1})^2}
\\
&+ \frac{2}{\Delta x^2}(\cos(k\Delta x)-1)\xi^ne^{ikj\Delta x} \frac{1}{1 + (
	D_1 \bar u^n_j
	)^2},
\end{split}
\end{align}
\label{ammc_von_neumann}
\end{subequations}
where $D_1\bar u^n_j = (\bar u^n_{j+1} -  \bar u^n_{j-1})/(2\Delta x)$.
All together, \eqref{ammc_von_neumann} then simplifies to 
\begin{align}
        \frac{\xi - 1}{\Delta t} 
= \frac{2}{\Delta x^2}\frac{\cos(k\Delta x)-1}{1 +(
	D_1\bar u^n_j 
	)^2}
+ \frac{1}{(\bar u_j^n)^2}
+ \frac{2}{\Delta x^2}(\cos(k\Delta x)-1)(\xi-1),
\end{align}
from which we can then isolate the amplification factor,
\begin{align}
        \xi 
= 1+\underbrace{\Delta t \frac{\frac{2}{\Delta x^2}\frac{\cos(k\Delta x)-1}{1 +(
	D_1\bar u^n_j 
	)^2}
+ \frac{1}{(\bar u_j^n)^2}}{1-\frac{2\Delta t}{\Delta x^2}(\cos(k\Delta x)-1)}}_{=w}.
\end{align}
Absolute stability, $\abs{\xi} < 1$, is attained, if and only if $-2<w<0$. First, $w<0$. As the denominator is positive, this will hold so long as the numerator is negative, 
\begin{align}
        \frac{2}{\Delta x^2}\frac{\cos(k\Delta x)-1}{1 +(
	D_1\bar u^n_j 
	)^2}
+ \frac{1}{(\bar u_j^n)^2} < 0 
\iff 
\left(\frac{\Delta x}{\bar u^n_j} \right)^2 
< \frac{2(1-\cos(k\Delta x))}{1 + (D_1\bar u^n_j)^2}.
\end{align}
This last relation is satisfied on the assumption that $\Delta x \ll \bar u^n_j$.  Next, we examine the numerator of $w+2$ in order to show $w+2 > 0$,
\begin{align}
\begin{split}
  \frac{2\Delta t}{\Delta x^2}\frac{\cos(k\Delta x)-1}{1 + (D_1 \bar u^n_j)^2} + \frac{\Delta t}{(\bar u^n_j)^2} + 2 - 4\frac{\Delta t}{\Delta x^2}(\cos(k\Delta x)-1) 
\\
= 2\frac{\Delta t}{\Delta x^2}(\cos(k\Delta x)-1)\left(
\frac{1}{1 + (D_1 \bar u^n_j)^2} - 2 \right)
+ 2 + \frac{\Delta t}{(\bar u^n_j)^2}.
\end{split}
\end{align}
Each term is positive without restriction and thus $\abs{\xi} < 1$ for all $\Delta t > 0$.

\subsection{Second order by Richardson extrapolation}



















\newpage
 
In this section, it will be important to give exact details as to the type of equations that I am solving. Then starting from a generic equation of this type, we add zero
\begin{align}
        u_t = \N(u)
= \underbrace{p\L u}_\text{linear} +\underbrace{ (\N(u) - p\L u)}_\text{nonlinear}, 
\quad 
p > 0.
\label{generic}
\end{align}

We ask for stability for all $\Delta t > 0$. What is a parabolic PDE? Elliptic PDE? More on the history, the result of Duchemin and Eggers \cite{duchemin2014explicit}.


We can also set up an example to convey our idea. Use the problem from Duchemin and Eggers \cite{duchemin2014explicit}: 
\begin{align*}
u_t &= \frac{u_{xx}}{1 + u_x^2} - \frac{1}{u}, 
\qquad 0 < x < 10, \, t > 0,
\end{align*}
with initial and boundary condition
\begin{align*}
u(x,0) &= 1 + 0.10\sin\left(\frac{\pi}{5} x \right),
\\
u(0,t) &= u(10,t) = 1.
\end{align*}
We will this problem throughout this thesis as testing grounds to demonstrate our ideas.

%\setkeys{Gin}{draft=false}
\chapter{IMEX Linear Multistep Methods}
For equations whose righthand side comprise of a stiff linear part and a nonstiff nonlinear component, a popular class of methods to apply are the implicit-explicit linear multistep methods (IMEX LMMs), with the most simple being an application of forward Euler to the nonlinear component and backward Euler to the linear, stiff component. 

\section{IMEX LMM Formulas}
In \cite{ascher1995implicit}, IMEX LMMs up to order four are investigated and a select number of schemes are singled out for their extensive use in the literature or for desired properties such as strong high frequency damping. As we are familiar with the first order variant, we begin by listing the second order methods of interest. These, and the higher order variants, will be presented as applied to the ODE 
\begin{align*}
u' = f + g, 
\end{align*}
where $f$ we identify as the nonlinear component and $g$ the stiff component. 
\\ \noindent
\textbf{Second order methods}\\
CNAB:
\begin{align}
\frac{u^{n+1}-u^n}{\Delta t} 
= \frac{3}{2} f^n - \frac{1}{2}f^{n-1} 
+ \frac{1}{2}(g^{n+1} + g^n), 
\label{cnab}
\end{align}\\
mCNAB:
\begin{align}
\frac{u^{n+1}-u^n}{\Delta t} 
= \frac{3}{2}f^n - \frac{1}{2} f^{n-1}
+ \frac{9}{16}g^{n+1} 
+ \frac{3}{8}g^n
+ \frac{1}{16}g^{n-1},
\label{mcnab}
\end{align}
\\
CNLF:
\begin{align}
\frac{u^{n+1}-u^{n-1}}{2\Delta t}
= f^n + \frac{1}{2}(g^{n+1} + g^{n-1}),
\label{cnlf}
\end{align} \\
SBDF2:
\begin{align}
\frac{3u^{n+1}-4u^n+u^{n-1}}{2\Delta t} 
= 2f^n - f^{n-1} + g^{n+1}.
\label{sbdf2}
\end{align}

\noindent
\textbf{Third order methods}\\
SBDF3:
\begin{align}
\frac{1}{\Delta t}\left(\frac{11}{6}u^{n+1} - 3u^n + \frac{3}{2}u^{n-1} - \frac{1}{3}u^{n-2} \right) 
= 3f^n - 3f^{n-1} + f^{n-2} + g^{n+1}.
\label{sbdf3}
\end{align}

\noindent
\textbf{Fourth order methods}\\
SBDF4:
\begin{align}
\frac{1}{\Delta t}\left(\frac{25}{12}u^{n+1} - 4u^n + 3u^{n-1} - \frac{4}{3}u^{n-2} + \frac{1}{4}u^{n-3} \right) 
= 4f^n - 6f^{n-1} + 4f^{n-2} -f^{n-3} + g^{n+1}.
\label{sbdf4}
\end{align}

Having given a collection of IMEX LMM formulas, we are now ready to apply such IMEX schemes to the modified test equation.

\section{Parameter Selection}
Let us restate here the modified test equation and some basic assumptions we make.
The modified test equation is 
\begin{align*}
        u' =  (1-p)\lambda u + p\lambda u.
\label{mtee}
\end{align*}
In our analysis, we will assume $p>0$ and $\lambda < 0$. The goal is then, for each IMEX LMM we consider, to identify a constraint on the parameter $p$, such that when satisfied, we may freely choose the time step-size without being subject to a stability restriction (eg. $\Delta t < C\Delta x^2$).

\subsection{Von Neumann Polynomials}
The resulting amplification polynomials from applying an $n$th order IMEX LMM to the modified test equation will be a degree $n$ polynomial in the amplification factor, $\xi$. The tool of choicie for the analysis of these amplification polynomials is the theory of Von Neumann polynomials \cite[Chapter 4]{strikwerda2004finite}. Below we give the relevant definitions and theorems.

\begin{definition}
        The polynomial $\phi$ is a von Neumann polynomial if all its roots, $r_q$, satisfy $\abs{r_q}\leq 1$.
\end{definition}
\begin{definition}
	The polynomial $\phi$ is a simple von Neumann polynomial is $\phi$ is a von Neumann polynomial and its roots on the unit circle are simple roots.
\end{definition}
\begin{definition}
	For any polynomial $\phi(\xi) = \sum^n_{j=0} a_j\xi^j$, we define the polynomial $\phi^*$ by $\phi^*(\xi) = \sum^n_{j=0} a^*_{n-j} \xi^j$, where $^*$ on the coefficient denotes the complex conjugate.
\end{definition}
\begin{definition}
	For any polynomial $\phi_n(\xi) = \sum^n_{j=0} a_j\xi^j$, we define recursively the polynomial $\phi_{n-1}$ by
	\begin{align}
	\phi_{n-1}(\xi) = \frac{\phi_d^*(0)\phi_d(\xi) - \phi_d(0)\phi_d^*(\xi)}{\xi}.
	\end{align} 
\label{defn:recurse}
\end{definition}
\begin{definition}
	$\phi$ is a Schur polynomial if all its roots, $r_q$, satisfy $\abs{r_q} < 1$.
\end{definition}
\begin{theorem}
	$\phi_n$ is a simple von Neumann polynomial if and only if either 
	\begin{enumerate}[label=(\alph{*})]
		\item $\abs{\phi_n(0)} < \abs{\phi^*_n(0)}$ and $\phi_{n-1}$ is a simple von Neumann polynomial or
		
		\item $\phi_{n-1}$ is identically zero and $\phi'_{n}$ is a Schur polynomial.
	\end{enumerate}
\label{thm:simple vN}
\end{theorem}

\subsection{Amplification polynomials of second order IMEX LMMs}
We start by applying CNAB \eqref{cnab} to the modified test equation \eqref{mte}. Combined with the anzatz $u^n = \xi^n$ and setting $z=\lambda\Delta t$, we get the amplification polynomial
\begin{align}
\Phi_2(\xi) 
= \left(1 - \frac{1}{2}zp\right)\xi^2
- \left(1 + z\left(\frac{3}{2}-p \right)\right)\xi + \frac{1}{2}z(1-p).
\label{amp_cnab}
\end{align}
The next series of steps will show that for $z<0$ (i.e. $\lambda < 0$,) \eqref{amp_cnab} is a simple von Neumann polynomial if and only if $p\geq 1$. This process involves showing that Theorem \ref{thm:simple vN} holds for $\Phi_2$.

We first give $\Phi_2^*$ and $\Phi_1$ as defined by the recursive process in Definition \ref{defn:recurse},
\begin{align}
\Phi_2^*(\xi) = \frac{1}{2}z(1-p) \xi^2 
- \left(1 + z\left(\frac{3}{2} - p \right)\right) \xi
+ 1 - \frac{1}{2}zp, 
\end{align}
and
\begin{align} 
\Phi_1(\xi) = \left(1 - zp + z^2 \left( \frac{1}{2} p - \frac{1}{4} \right) \right)\xi -1 - z(1-p) + z^2 \left( \frac{3}{4} - \frac{1}{2}p \right).
\end{align}
Next, we verify that if $p\geq 1$, then $\abs{\Phi_2(0)} < \abs{\Phi^*_2(0)}$. Reformulating the expression as
\begin{align*}
\abs{\Phi_2(0)} < \abs{\Phi^*_2(0)} 
\iff 
0 < \left(\Phi^*_2(0) \right)^2 - \Phi_2(0)^2
= 1 - zp - \frac{1}{2}z^2\left( \frac{1}{2} - p \right),
\end{align*}
(and keeping in mind that we ask this inequality to hold only for $z<0$,) we find that the contribution from each term  in the rightmost quadratic is positive, thus establishing the claim.
Finally, we show that $\Phi_1$ is simple von Neumann directly. Denoting the root of $\Phi_1$ as $\xi_1$, 
\begin{align*}
	\abs{\xi_1} < 1 
\iff \abs{\frac{(2p-3)z-2}{(2p-1)z-2}} < 1
\iff 0< 8z((p-1)z-1),  
\end{align*}
which holds for all $z<0$ if and only if $p\geq 1$.

Other IMEX LMMs are analyzed in the same way. Their amplification polynomials are recorded for reference in Table \ref{table:amp poly 2}, along with the parameter range for which we observe unconditional stability.

\begin{table}[htb!]
	\centering
	\caption{Amplification polynomial and choice of parameter when applying second order IMEX LMMs to the modified test equation for unconditional stability.}
	\begin{tabular}{lll}
		\toprule[1.5pt] 
		\head{Method} 
		& \head{Amplification Polynomial}
		& \head{$p\lambda/\lambda\in$}
		\\	\midrule 
		CNAB 
		& $\left(1 - \frac{1}{2}zp\right)\xi^2
		- \left(1 + z\left(\frac{3}{2}-p \right)\right)\xi + \frac{1}{2}z(1-p)$
		& $[1,\infty)$
		\\[2.6pt]
		mCNAB 
		& $\left(1 - \frac{9}{16}zp \right) \xi^2 - \left(1 + z\left(\frac{3}{2} - \frac{9}{8} p \right) \right)\xi
		+ \frac{1}{2}z\left(1 - \frac{9}{8}p \right) $
		& $[8/9,\infty)$
		\\[2.6pt]
		CNLF 
		& $\left(1-pz\right) \xi^2 -2z(1-p)\xi -(1+pz)$
		& $[1/2,\infty)$
		\\[2.6pt]
		SBDF2 
		& $\left(\frac{3}{2} - zp\right) \xi^2
		- 2\left(1 + z(1-p)\right) \xi 
		+ \frac{1}{2} + z(1-p)
		$
		& $[3/4,\infty)$
		\\ \bottomrule[1.5pt]
	\end{tabular}
\label{table:amp poly 2}
\end{table}

\subsection{Amplification polynomials of third and fourth oder IMEX LMMs}
We continue with this type of analysis for higher order IMEX LMMs. Again, amplification polynomials and parameter ranges are derived, but because the expressions and manipulations quickly become cumbersome and tedious for higher order methods, the computer algebra system, \textsc{Maple}, was used to for the majority of the calculations.

\begin{table}[htb!]
	\centering
	\caption{Amplification polynomial and choice of parameter when applying high order IMEX LMMs to the modified test equation for unconditional stability.}
	\begin{tabular}{lll}
		\toprule[1.5pt] 
		\head{Method} 
		& \head{Amplification Polynomial}
		& \head{$p\lambda/\lambda\in$}
		\\	\midrule 
		SBDF3
		& $\left(\frac{11}{6} - zp \right)\xi^3
		- 3\left(1 + z(1-p) \right) \xi^2 
		+ \frac{3}{2}\left(1 + 2z(1-p) \right) \xi 
		$
		& $[7/8, 2]$
		\\
		& \phantom{$\left(\frac{11}{6} - zp \right)\xi^3
			- 3\left(1 + z(1-p) \right) \xi^2$}$- \frac{1}{3}\left(1 +  3z(1-p) \right)$
		\\ [2.6pt]
		SBDF4
		& $\left(\frac{25}{12} - zp \right) \xi^4
		- 4\left(1 + z(1-p) \right)\xi^3
		+ 3\left(1 + 2z(1-p) \right)\xi^2 
		$
		& $[15/16, 5/4]$ 
		\\
		& \phantom{$\left(\frac{25}{12} - zp \right) \xi^4$}$- \frac{4}{3}\left(1 + 3z(1-p) \right) \xi
		+ \frac{1}{4}\left(1 + 4z(1-p)\right)$
		\\ \bottomrule[1.5pt]
	\end{tabular}
	\label{table:amp poly 34}
\end{table}

Listed in Table \ref{table:amp poly 34} are respectively the amplification factor and the parameter restriction for SBDF3 and SBDF4. A crucial difference when comparing these higher order methods to the second order methods is that the restriction leaves only a finite interval. This will be addressed further in the subsequent section. In fact, we will demonstrate that this detail renders the linearly stabilized SBDF3, SBDF4 useless in practical application.

\subsection{Von Neumann stability analysis on a 1D test problem}
We now take our theory on a test drive with the 1D mean curvature motion test problem from before. Recall that after discretization in space by second order centred differences, we have the system of ODEs
\begin{align*}
\dd{U_j}{t} 
= 4\frac{U_{j+1} - U_j + U_{j-1}}{4\Delta x^2 + (U_{j+1} - U_{j-1})^2} 
- \frac{1}{U_j}, 
\quad 
j=1:N,
\end{align*} 
along with $U_0 = U_{N+1} = 1$.

\setkeys{Gin}{draft}
\chapter{Higher Order with Exponential Integrators}
The investigation with multistep schemes left us with a major question. Since the linearly stabilized schemes derived from SBDF3, SBDF4 were shown to be unsuitable for practical use, is it possible to construct practical high order linearly stabilized time stepping methods? In this chapter, we consider two methods coming from the class of exponential integrators. We will demonstrate that the second and fourth order exponential Runge-Kutta from Cox and Matthews \cite{cox2002exponential} work well within our linear stabilization framework.

\section{Exponential Runge-Kutta}
\label{sect:exp rk}
Consider the ODE
\begin{align}
\dd{u}{t} = \N(u) + \L u.
\end{align}
Exponential time differencing, or exponential integrators, is a family of time stepping methods that treats the linear part exactly, and approximates the nonlinear part by some suitable quadrature formula. As an example, the exponential Euler method has the formula
\begin{align}
u^{n+1} 
= e^{\Delta t\L} u^n + \L^{-1}(e^{\Delta t\L}  -1) \N(u^n).
\end{align}
This is a first order accurate exponential integrator. 
 
Our investigation covers explicit exponential Runge-Kutta methods only. This family of one-step methods have the form 
\begin{subequations}
	\begin{align}
u^{n+1} &= e^{\Delta t \L} u_n 
+ \Delta t \sum^s_{i=1} b_i(\Delta t\L)\N(U^{n,i}) 
\\
U^{n,i} &= e^{c_i\Delta t \L} u^n 
+ \Delta t\sum^{i-1}_{j=1} a_{ij}(\Delta t \L) \N(U^{n,j}),
	\end{align}
	\label{exp rk general}
\end{subequations}
and can be presented in the familiar Butcher tableau:
\newcommand\raisepunct[1]{\,\mathpunct{\raisebox{-5.20ex}{#1}}}
\begin{align}
\begin{tabular}{c|cccc}
$c_1$
&  
\\
$c_2$ & $a_{21}$ & 
\\
$\vdots$ & $\vdots$ & $\ddots$ 
\\
$c_s$ & $a_{s1}$ & $\cdots$ & $a_{s,s-1}$ 
\\ \hline 
& $b_1$ & $\cdots$ & $b_{s-1}$ & $b_s$
\end{tabular}\raisepunct{.}
\end{align}
Note that we have suppressed the argument, but these are indeed functions of $\Delta t\L$.
In particular, we focus on the second and fourth order exponential Runge-Kutta formulas of Cox and Matthews \cite{cox2002exponential};
\renewcommand\raisepunct[1]{\,\mathpunct{\raisebox{-1.30ex}{#1}}}
\begin{align} 
\begin{tabular}{c|cc}
$0$ 
& 
\\
\num{1} 
& $\varphi_{1,2}$ 
&
\\ \hline
& $\varphi_1 - \varphi_2$ 
& $\varphi_2$
	\end{tabular}\raisepunct{,}
\label{etdrk2 butcher}
\end{align}
\renewcommand\raisepunct[1]{\,\mathpunct{\raisebox{-4.80ex}{#1}}}
\begin{align} 
\begin{tabular}{c|cccc}
$0$ 
&  
\\
\num{1/2} 
& $\frac{1}{2}\varphi_{1,2}$ 
&
\\ 
\num{1/2} 
& 0 
& $\frac{1}{2}\varphi_{1,3}$ 
& 
\\
\num{1} 
& $\frac{1}{2}\varphi_{1,3}(\varphi_{0,3}-1)$ 
& 0 
& $\varphi_{1,3}$ 
& 
\\ \hline 
& $\varphi_1 - 3\varphi_2 + 4\varphi_3$ 
& $2\varphi_2 - 4\varphi_3$ 
& $2\varphi_2 - 4\varphi_3$ 
& $4\varphi_3 - \varphi_2$ 
\end{tabular}\raisepunct{,}
\label{etdrk4 butcher}
\end{align}
where 
\begin{align}
        \varphi_{k+1}(z) = \frac{\varphi_k(z) - 1/k!}{z}, 
\quad \varphi_0(z) = \exp(z), 
\qand 
\varphi_{i,j}(z) = \varphi_{i}(c_j z).
\label{phi functions}
\end{align}
We refer to this pair of exponential Runge-Kutta methods as ETDRK2 and ETDRK4, respectively.

\section{Linearly stabilized ETDRK2 and ETDRK4}
In the last chapter, we identified some useful criteria for quickly assessing the practicality of newly constructed linearly stabilized methods. 

First, we apply the schemes \cref{etdrk2 butcher,etdrk4 butcher} to the modified test equation \cref{mte} and imposed unconditional stability. The resulting parameter restriction must be unbounded, otherwise we stop. For ETDRK2 and ETDRK4, with the help of the computer algebra system, \textsc{Maple}\textsuperscript{\texttrademark}, we determined the parameter restriction to be $[1/2, \infty)$ in both cases. In \cref{fig:etdrk2 stab,fig:etdrk4 stab} are stability contour plots that support this claim.

\begin{figure}[htb!]
	\centering
%\includegraphics[width=0.65\textwidth]{etdrk2_stab}
\includegraphics[width=0.65\textwidth]{etdrk2_stab2}
\caption{Stability contours for ETDRK2 at various $p$.}
\label{fig:etdrk2 stab}
\end{figure}

\begin{figure}[htb!]
	\centering
%\includegraphics[width=0.65\textwidth]{etdrk4_stab}
\includegraphics[width=0.65\textwidth]{etdrk4_stab2}
\caption{Stability contours for ETDRK4 at various $p$.}
\label{fig:etdrk4 stab}
\end{figure}

We then also considered two other factors that affected the performance. The first is a proxy to the error constant of the numerical scheme, and the second is to consider the amplification factor at infinity. The latter is plotted in \cref{fig:damp fac at inf etd} for both ETDRK2 and ETDRK4. It shows that the ETDRK schemes provide strong damping at infinity for a wide range of $p$ and may be a good candidate for taking large time steps. For the proxy to the error constant, what we did before was we took the amplification factors and found its series expansion at $z=0$,
\begin{align}
        \xi_{ETDRK2} = 1+ z + \frac{z^2}{2} + \left(-\frac{1}{4}p^2 + \frac{5}{12}p \right) z^3 + \cdots ,
\end{align}
\begin{align}
        \xi_{ETDRK4} = \exp(z) + \left( 
\frac{1}{576}p^4 
- \frac{11}{576}p^3
+ \frac{29}{720}p^2
- \frac{1}{32}p
+ \frac{1}{120} 
\right) z^5 + \cdots .
\end{align}
Let us take this one step further by taking the difference with the exact solution, $\exp(z)$. We find 
\begin{align}
        \delta_1(p) = \abs{\exp(z) - \xi_{ETDRK2}} 
= \abs{\frac{1}{4}p^2 - \frac{5}{12}p + \frac{1}{6}},
\end{align}
\begin{align}
        \delta_2(p) = \abs{\exp(z) - \xi_{ETDRK4}} 
= \abs{\frac{1}{576}p^4 
- \frac{11}{576}p^3
+ \frac{29}{720}p^2
- \frac{1}{32}p
+ \frac{1}{120} },
\end{align}
and for the EIN method we have 
\begin{align}
\delta_3(p) = \abs{\exp(z) - \xi_{EIN}} 
= \abs{\frac{1}{2} p^2 - \frac{1}{2}p + \frac{1}{6}}.
\end{align}

\begin{figure}[htb!]
        \centering
\includegraphics[width=0.65\textwidth]{damp_at_inf_2}
\caption[Amplification factors at infinity with ETDRK schemes.]{Amplification factors at infinity with ETDRK schemes. Included also for the purpose of comparison are some second order schemes from the last chapter. The normalization along the horizontal axis is with respect to the lower limit of the parameter restriction of each scheme.}
\label{fig:damp fac at inf etd}
\end{figure}

Recall from before that the observed convergence rate of EIN was reduced and we reasoned that that was because the error constant is large whenever $p$ is large, and thus the step-size had to be chosen much smaller than desirable before observing second order convergence.

The situation is similar with the ETDRK schemes. The error term has quadratic and quartic polynomials in $p$ for ETDRK2 and ETDRK4 respectively, and thus we expect these schemes to fair poorly when $p$ is large. That being said, we do find in \cref{chap:num experiments} that these scheme can outperform SBDF2 and CNAB when $p$ is small and the step-size is coarse. To this end, we provide \cref{fig:err p dep} that shows $\delta_1$, $\delta_2$, and $\delta_3$ plotted against $p$. Also plotted are the analogous expressions for SBDF2 and CNAB. As was suggested, there may be a range of $p$ for which these ETDRK schemes are workable. A careful error analysis on fully nonlinear problems would be of some interest and is left as future work.

\begin{figure}[htb!]
        \centering
\includegraphics[width=0.65\textwidth]{err_const}
\caption[Error constant and its dependence on $p$.]{Error constant and its dependence on $p$. }
\label{fig:err p dep}
\end{figure}

\subsection{Stable evaluation of the matrix exponential and related functions}
Finally, we wish to discuss briefly the matter of implementing ETDRK schemes.

In \cref{sect:exp rk}, we have presented two exponential Runge-Kutta methods in tableau form and the entries of these tableaus are functions of the operator, $\Delta t\L$.  For example, to implement ETDRK2, we must compute the functions 
\begin{align}
\varphi_1(c\Delta \L) &= (c\Delta t \L)^{-1}(\exp(c\Delta t \L) - 1),
\\
\varphi_2(c\Delta \L) &= (c\Delta t\L)^{-2}(\exp(c\Delta t \L)  -1 - c\Delta t \L) ,
\end{align}
for $c \in [0,1]$, or be able to efficiently evaluate the related matrix-vector multiplications without explicit construction.

Difficulties with the evaluation of the matrix exponential and related functions of the form \cref{phi functions} are well-known and well-studied, eg.\ \cite{moler2003nineteen,higham2002accuracy,higham2008functions,hochbruck1997krylov}. Thus widespread adoption of exponential time differencing methods must first contend with the developement of stable and efficient algorithms for computing the matrix exponential. 

In our examples, we follow the direction of Kassam and Trefethen \cite{kassam2005fourth}. They take a contour integral approach to the evaluation of functions in the form of \cref{phi functions} coupled with the trapezoidal rule for fast, accurate, and stable computations. Moreover, to avoid the unpleasantness of boundary conditions (and the subsequent complications), our domain will be periodic when using ETDRK2 or ETDRK4. 

Lastly, let us mention that there are other popular approaches. Amongst these are those based on scaling and squaring, Pad\'{e} approximants \cite{moler2003nineteen,higham2008functions}, and Krylov subspace methods  \cite{hochbruck1997krylov,sidje1998expokit,simoncini2007recent}.



\chapter{Numerical Experiments}
\label{chap:num experiments}
This chapter is entirely devoted to solving stiff PDEs prevalent in a number of fields. The experiments we demonstrate will fall under three categories: image inpainting, interface motion, and phase separation. For each problem, we will give the PDE model and then discuss how we stabilize and select the parameters. Our experiments will show the feasibility of these schemes in 2D and 3D. 

Before we proceed further, we would like to make a few notes on our implementation of these schemes. As stated from the outset, our goal is to provide simple, accurate, and efficient time stepping methods for nonlinear PDEs. Too often, new efficient methods are so complicated that the user resorts to simple explicit schemes known to require lengthy computing time rather than invest an indeterminate amount of time understanding, implementing, and debugging.

In our implementation, the choice of $p$ is fixed throughout the time evolution. This is not the only way as one may want to adapt $p$ as the solution evolves. This could be advantageous as overestimates of $p$ lead to larger errors. However, we do not pursue that here. So while our theory speaks of approximating the eigenvalues of the linearized system, we do not incur this cost. 

Moreover, a static value of $p$ offers the advantage that the linear system to be solved is the same at each time step, i.e.\ the matrix to be inverted is static. Any expensive preprocessing/factorizing of this matrix needs only be done once.

\section{Image Inpainting}
Image inpainting is the task of repairing corrupted images and damaged artwork \cite{bertalmio2000image}. In the inpainting examples to follow, the user identifies in the image the region to be inpainted, and from there, a PDE model is evolved to fill-in the inpainting region using the neighbouring information.

Two PDE models are selected. The first is a second order model from Shen and Chan \cite{shen2002mathematical}, and the second is a recent fourth order model from Sch{\"o}nlieb and Bertozzi \cite{schonlieb2011unconditionally},
\begin{align}
        u_t  = \nabla \cdot \left(\frac{\nabla u}{\sqrt{\abs{\nabla u}^2 + \epsilon^2}} \right) 
+ \lambda_D(u_0 - u), 
\label{bv inpaint}
\end{align}
\begin{align}
         u_t  = -\Delta \nabla \cdot \left(\frac{\nabla u}{\sqrt{\abs{\nabla u}^2 + \epsilon^2}} \right) 
+ \lambda_D(u_0 - u).
\label{tvhneg inpaint}
\end{align}
We refer to these as TV inpainting and TV-H$^{-1}$ inpainting.

The solution, $u$, is the inpainted image we seek. The quantity, $u_0$, is the initially corrupted image, and $\epsilon > 0$ is a regularization parameter. Denoting the image domain $\Omega$, and the inpainting region, $D$, we then define $\lambda_D$ as 
\begin{align}
\lambda_D(x)
= \begin{cases}
\lambda_0, & x\in \Omega\setminus D
\\
0, &\text{otherwise},
\end{cases}
\end{align}
for some $\lambda_0 > 0$.

As for initial conditions, we have two vandalized images to be restored. The first is \cref{fig:turtle with text} where we have a photograph of a sea turtle overwritten with text. We would like to restore the photograph by removing the text. The second is \cref{fig:bullfinch and fox}, where the fox figure requires removal. Although this may look simpler, it is in fact a more challenging scenario. The reason being that the thickness of the inpainting region requires correctly extending level lines over long distances \cite{schonlieb2011unconditionally}.

The image is restored in a channel-by-channel manner. As a stopping criteria, we look at the difference in successive images, 
\begin{align}
        \frac{\norm{u^{n+1} - u^n}_2}{\norm{u^{n+1}}_2} < \mathrm{tol},
\end{align}
where the tolerance is set at $\mathrm{tol}=\num{9e-5}$.  In addition, we also set a maximum of 500 iterations per channel. We report the time step-size and the total number of iterations used for each scheme.

Finally, we note that the spatial discretization is by second order centred differences with uniform spacing, $h$, in both $x$ and $y$.
\begin{figure}[htb!]
        \centering 
\includegraphics[width=0.65\textwidth]{turtle_with_text}
\caption[Photograph of turtle overwritten with text.]{Photograph of a sea turtle overwritten with text.}
\label{fig:turtle with text}
\end{figure}

\begin{figure}[htb!]
        \centering 
\includegraphics[width=0.65\textwidth]{bullfinch_and_fox}
\caption[Photograph of a bullfinch vandalized by a cartoonish fox.]{Photograph of a bullfinch vandalized with a cartoon fox.}
\label{fig:bullfinch and fox}
\end{figure}

\subsection{TV inpainting}
We first show that the TV inpainting model can easily be handled by our methods. In this model, there are two terms on the right-hand side, both potentially stiff.  The second term is stabilized by adding and subtracting $-p_2\lambda_0 u$, and we determined $p_2=p_0$, where $p_0$ is the minimum value required for unconditional stability that we find from applying the time stepping method to the modified test equation. For the first term we stabilize by adding and subtracting $p_1\Delta u$. To determined $p_1$, we first bound the first term as 
\begin{align}
\begin{split}
        \nabla \cdot \left(\frac{\nabla u}{\sqrt{\abs{\nabla u}^2 + \epsilon^2}} \right) 
&= \frac{u_{xx}(u_y^2 + \epsilon^2) + u_{yy}(u_x^2 + \epsilon^2)}{(u_x^2 + u_y^2 + \epsilon^2)^{3/2}} 
- \frac{2u_xu_y u_{xy}}{(u_x^2 + u_y^2 + \epsilon^2)^{3/2}} 
\\
&\leq \frac{(u_{xx}+u_{yy})(u_x^2 + u_y^2 + \epsilon^2)}{(u_x^2 + u_y^2 + \epsilon^2)^{3/2}} + \frac{(u_x^2 + u_y^2 + \epsilon^2)u_{xy}}{(u_x^2 + u_y^2 + \epsilon^2)^{3/2}}  
\\
&= \frac{u_{xx} + u_{yy} + u_{xy}}{\sqrt{u_x^2 + u_y^2 + \epsilon^2}}.
\end{split}
\label{p1 estimate inpaint 1}
\end{align}
We then consider the auxiliary equation $u_t = u_{xx} + u_{yy} + u_{xy}$ discretized by centred differences in space and forward Euler in time to apply a von Neumann analysis with $u^n_{jk} = \xi^n \exp(i\omega_1jh)\exp(i\omega_2kh)$ to get 
\begin{align}
        \frac{\xi  - 1}{\Delta t} 
= \frac{1}{h^2}(-4 + 2\cos(\omega_1 h) + 2\cos(\omega_2 h) - \sin(\omega_1 h)\sin(\omega_2h))
\geq -\frac{8}{h^2}.
\label{p1 estimate inpaint 2}
\end{align}
Combined with the assumption of the extreme case, $\sqrt{u_x^2 + u_y^2 + \epsilon^2} \geq \epsilon$, we set $p_2$ according to 
\begin{align}
        p_1\frac{8/h^2}{8/(\epsilon h^2)} \geq p_0 \iff 
p_1 \geq \frac{1}{\epsilon}p_0.
\end{align}

The images restored by TV inpainting are shown in \cref{fig:bv inpainting}. We restored using SBDF1, SBDF2, and CNAB for time stepping. Parameters were set as $\epsilon=0.10$ and $\lambda_0=20$. \cref{tab:bv iter counts} charts the least iteration count we achieved with each method. The numbers show a definite improvement of using the second order methods over the first order. 
\begin{figure}[htb!]
\centering
\begin{minipage}{0.65\textwidth}
	\includegraphics[width=\textwidth]{bv_turtle}
\end{minipage}
\begin{minipage}{0.65\textwidth}
	\includegraphics[width=\textwidth]{bv_bullfinch}
\end{minipage}
\caption[Image restoration by TV inpainting.]{Image restoration by TV inpainting using a second order linearly stabilized time stepping method.}
\label{fig:bv inpainting}
\end{figure}

% \begin{figure}[htb!]
%         \centering
% \includegraphics[width=0.65\textwidth]{bv_bullfinch_s1}
% \caption[Image restoration by TV inpainting -- SBDF1.]{Image restoration by TV inpainting using SBDF1.}
% \label{fig:bv bullfinch s1}
% \end{figure}

\begin{table}[htb!]
\caption[Iteration counts for TV image restoration.]{Iteration counts for TV image restoration.}
        \centering\begin{tabular}{lll ll} \toprule[1.25pt]
& \multicolumn{2}{c}{Sea Turtle} & \multicolumn{2}{c}{Bullfinch}
\\
& $\Delta t$ & Iterations & $\Delta t$ & Iterations
\\ \midrule
SBDF1 & 0.50 & 141 & 0.33 & 393 
\\
SBDF2& 0.18 & 115 & 0.22 & 168
\\             
CNAB &  0.16 & 104 & 0.18 & 195
\\ \bottomrule[1.25pt]
\end{tabular}
\label{tab:bv iter counts}
\end{table}

\subsection{TV-H\texorpdfstring{$^{-1}$}{-1} inpainting}
For TV-H$^{-1}$ inpainting, we stabilize \cref{tvhneg inpaint} as 
\begin{align}
        u_t = -\Delta \nabla \cdot \left(\frac{\nabla u}{\sqrt{\abs{\nabla u}^2 + \epsilon^2}} \right) + \lambda_D(u_0 - u)  + p_1\Delta^2 u + p_2\lambda_0 u - p_1\Delta^2 u - p_2\lambda_0 u.
\label{tvhneg stabilized}
\end{align}
We note $p_2$ is chosen exactly as in TV inpainting. Determination of $p_1$ makes use of the bound set in \cref{p1 estimate inpaint 1,p1 estimate inpaint 2} to get 
\begin{align}
        p_1\frac{(8/h^2)^2}{(8/h^2)(8/(\epsilon h^2))} \geq p_0 
\iff p_1 \geq \frac{1}{\epsilon}p_0,
\end{align}
which is also the same as TV inpainting.

Interestingly, in the same paper where they propose \cref{tvhneg inpaint} for image inpainting, they offer exactly \cref{tvhneg stabilized} and time stepping with SBDF1 as the solution algorithm. In \cref{tab:bvhneg iter counts}, we chart again the least iteration counts achieved with each of SBDF1, SBDF2, CNAB, with parameters $\epsilon=0.10$ and $\lambda_0=30$. Once again, the second order methods are an improvement over the first order method, with the improvement especially notable in the more difficult example with the bullfinch. 

As a final note, we would like to mention other interesting developments in \cite{bredies2010total,papafitsoros2014combined,papafitsoros2013combined}. In these papers, they propose a number of image restoration models involving a number of high order derivatives interacting nonlinearly. It would be of some interest to compare the effectiveness of our schemes with the methods they propose.

\begin{figure}[htb!]
	\centering
\begin{minipage}{0.65\textwidth}
	\includegraphics[width=\textwidth]{bvhneg_turtle}
\end{minipage}
\begin{minipage}{0.65\textwidth}
	\includegraphics[width=\textwidth]{bvhneg_bullfinch}
\end{minipage}
\caption[Image restoration by TV-H$^{-1}$ inpainting.]{Image restoration by TV-H$^{-1}$ inpainting using a second order linearly stabilized time stepping method.}
\label{fig:bvhneg inpainting}
\end{figure}

\begin{table}[htb!]
\caption[Iteration counts for TV-H$^{-1}$ image restoration.]{Iteration counts for TV-H$^{-1}$ image restoration.}
        \centering\begin{tabular}{lll ll} \toprule[1.25pt]
& \multicolumn{2}{c}{Sea Turtle} & \multicolumn{2}{c}{Bullfinch}
\\
& $\Delta t$ & Iterations & $\Delta t$ & Iterations
\\ \midrule
SBDF1 & 0.65 & 143 & 0.30 & 1002
\\
SBDF2& 0.12 & 119 & 0.54 & 401 
\\             
CNAB & 0.10 & 108 & 0.64 & 347
\\ \bottomrule[1.25pt]
\end{tabular}
\label{tab:bvhneg iter counts}
\end{table}


\section{Motion by Mean Curvature}
In this section, we study the problem of interface evolution under mean curvature flow. The level set equation for motion by mean curvature is 
\begin{align}
        u_t 
= \kappa\abs{\nabla u} 
= \abs{\nabla u} \nabla \cdot \left(\frac{\nabla u}{\abs{\nabla u}} \right).
\label{mcm}
\end{align}
Our interest is in the time evolution of the interface, $\Gamma=\Gamma(t)$, described by the zero level set of the function $u$,
\begin{align}
        \Gamma(t) = \{x \in \bR^d \mid u(x,t) =0\}.
\end{align}

We will demonstrate the effectiveness of our schemes on examples similar to that of Smereka \cite{smereka2003semi}. In \cite{smereka2003semi}, he uses what we call linearly stabilized SBDF1 to take large, stable time steps. In fact, in the same paper it was also suggested that Richardson extrapolation may be used to attain second order convergence, but was not implemented. We follow the example set in \cite{smereka2003semi} and stabilize \cref{mcm} with a Laplacian term, $p\Delta u$, to get 
\begin{align}
        u_t = \kappa \abs{\nabla u} - p\Delta u + p\Delta u.
\end{align}
An analysis similar to \cref{p1 estimate inpaint 1,p1 estimate inpaint 2} yields $p\geq p_0$ to be sufficient for unconditional stability.

Let us point out a key difference between this problem and the inpainting problem of the previous section. In the inpainting problem, the system was to be driven to steady state. As such, we were afforded a range of time step-sizes where the solution method performed well in terms of speed. Visually, the step-size did not affect the accuracy. For mean curvature flow, computing time and accuracy are directly related to the choice of step-size. Thus we are interested in large step-sizes only if an acceptable level of accuracy is maintained.

\subsection{Shrinking dumbbell in 2D and 3D}
Our first example is evolution of a dumbbell-shaped curve in 2D. \cref{fig:mcm 2d dumbbell} plots the motion of this curve under mean curvature flow. From the initial dumbbell shape, the corners smooth out and then the curve shrinks as shown, until it eventually collapses down to a point. These were generated with an explicit Runge-Kutta method solving to time $T=1.25$. On a periodic grid of size $256\times 512$ and approximating the spatial derivatives using second order centred differences, the number of time steps needed for stability is $\O(10^4)$.

\begin{figure}[htb!]
        \centering
\begin{minipage}{0.48\textwidth}
        \includegraphics[width=0.96\textwidth]{dumbbell_0}
\end{minipage}%
\begin{minipage}{0.48\textwidth}
        \includegraphics[width=0.96\textwidth]{dumbbell_001}
\end{minipage}
\begin{minipage}{0.48\textwidth}
        \includegraphics[width=0.96\textwidth]{dumbbell_05}
\end{minipage}%
\begin{minipage}{0.48\textwidth}
        \includegraphics[width=0.96\textwidth]{dumbbell_125}
\end{minipage}%
\caption[Mean curvature flow of a dumbbell-shaped curve in 2D.]{Mean curvature flow of a dumbbell-shaped curve in 2D. From the top left to the bottom right, the plots show the evolution at times $t=0$, $0.01$, $0.5$, $1.25$.}
\label{fig:mcm 2d dumbbell}
\end{figure}

In \cref{fig:mcm 2d conv}, we show the convergence of SBDF1, SBDF2, CNAB, EIN, ETDRK2, ETDRK4 to the same problem, but plotting in the positive quadrant only. As shown, each scheme is stable and convergent  using a number of time steps much smaller than $10^4$. However, it is important to note that amongst the schemes, the number of time steps needed to achieve an acceptable level of accuracy ranges from 50 to 800. 

\begin{figure}[htb!]
        \centering
\begin{minipage}{0.48\textwidth}
        \includegraphics[width=0.96\textwidth]{s1_conv}
\end{minipage}
\begin{minipage}{0.48\textwidth}
        \includegraphics[width=0.96\textwidth]{ein_conv}
\end{minipage}
\begin{minipage}{0.48\textwidth}
        \includegraphics[width=0.96\textwidth]{s2_conv}
\end{minipage}
\begin{minipage}{0.48\textwidth}
        \includegraphics[width=0.96\textwidth]{erk2_conv}
\end{minipage}
\begin{minipage}{0.48\textwidth}
        \includegraphics[width=0.96\textwidth]{cnab_conv}
\end{minipage}
\begin{minipage}{0.48\textwidth}
        \includegraphics[width=0.96\textwidth]{erk4_conv}
\end{minipage}
\caption[Convergence of linearly stabilized schemes to a shrinking dumbbell problem.]{Convergence of various linearly stabilized time stepping methods to the mean curvature flow of a dumbbell-shaped curve in 2D. In the column to the left from top to bottom, we have SBDF1, SBDF2, and CNAB. In the column to the right from top to bottom, we have EIN, ETDRK2, and ETDRK4. Within each plot, each curve is the solution using the number of time steps, $n$, as indicated.}
\label{fig:mcm 2d conv}
\end{figure}

The most disappointing of these is CNAB, for which a minimum of 400 time steps was needed to get a solution that goes with the frame we have chosen and further needs close to 800 time steps to present a solution competitive with the other methods in accuracy. SBDF2 likewise fared poorly in that regards, needing close to 800 time steps for a solution competitive with the other methods in accuracy. 

On the other hand, the ETDRK schemes and the EIN method appears to do well at even 50-100 time steps. Recall that these are the schemes that have strong damping at infinity (see \cref{fig:damp fac at inf etd}). Zoom-ins to the solution by these methods are presented in \cref{fig:mcm zoom in 1,fig:mcm zoom in 2}. These show that the ETDRK schemes offer good accuracy and convergence at just 50 time steps. The EIN method, however, is again hampered by slow convergence. This is especially clear in \cref{fig:mcm zoom in 1}. If we further factor in computing time, then both CNAB and SBDF2 would outperform the EIN method.

\begin{figure}[htb!]
        \centering
\includegraphics[width=0.65\textwidth]{ein_zoom1}
\includegraphics[width=0.65\textwidth]{erk2_zoom1}
\includegraphics[width=0.65\textwidth]{erk4_zoom1}
\caption[{A zoom-in over $ [0.0,0.20]\times [0.475,0.54]$ to inspect convergence.}]{A zoom-in over $[0.0, 0.20]\times [0.475, 0.54]$. From top to bottom we have the EIN method, ETDRK2, and ETDRK4. We see slow convergence of the EIN method, and good convergence of the ETDRK schemes.}
\label{fig:mcm zoom in 1}
\end{figure}

\begin{figure}[htb!]
        \centering
\begin{minipage}{0.30\textwidth}
        \includegraphics[width=0.92\textwidth]{ein_zoom2}
\end{minipage}
\begin{minipage}{0.30\textwidth}
        \includegraphics[width=0.92\textwidth]{erk2_zoom2}
\end{minipage}
\begin{minipage}{0.30\textwidth}
        \includegraphics[width=0.92\textwidth]{erk4_zoom2}
\end{minipage}
\caption[{A zoom-in over $[1.30,1.41]\times [0.0, 0.30]$ to inspect convergence.}]{A zoom-in over $[1.30,1.41]\times [0.0, 0.30]$. From left to right we have the EIN method, ETDRK2, and ETDRK4. We see slow convergence of the EIN method, and good convergence of the ETDRK schemes.}
\label{fig:mcm zoom in 2}
\end{figure}

Next, we take this example into 3D. Here, the advantages of our schemes are further magnified. In 2D, one could argue that the computations can be completed using standard explicit schemes within reasonable computing times. In 3D, doing so may require trade offs in the grid size, or computing only over short time periods. 

Setting the initial condition to be the dumbbell-shaped curve of the top left image in \cref{fig:mcm 3d dumbbell}, the curve is then evolved by mean curvature flow. We use a periodic grid of size $256\times 128\times 128$ and solve to time $T=0.75$. With forward Euler, we needed $3000$ time steps for stability, and this computation took over 28 minutes in \textsc{Matlab} 2014b on an Intel\textsuperscript{\textregistered}Core\textsuperscript{\texttrademark}i5-4570 CPU@3.20GHz workstation running Linux. With the linearly stabilized ETDRK2, we solved the same problem using 75 time steps in 103 seconds.

\begin{figure}[htb!]
        \centering
\begin{minipage}{0.48\textwidth}
        \includegraphics[width=0.96\textwidth]{dumbbell_3d0}
\end{minipage}%
\begin{minipage}{0.48\textwidth}
        \includegraphics[width=0.96\textwidth]{dumbbell_3d01}
\end{minipage}
\begin{minipage}{0.48\textwidth}
        \includegraphics[width=0.96\textwidth]{dumbbell_3d03}
\end{minipage}%
\begin{minipage}{0.48\textwidth}
        \includegraphics[width=0.96\textwidth]{dumbbell_3d0525}
\end{minipage}
\begin{minipage}{0.48\textwidth}
        \includegraphics[width=0.96\textwidth]{dumbbell_3d055}
\end{minipage}%
\begin{minipage}{0.48\textwidth}
        \includegraphics[width=0.96\textwidth]{dumbbell_3d075}
\end{minipage}
\caption[Mean curvature flow of a dumbbell-shaped curve in 3D.]{Mean curvature flow of a dumbbell-shaped curve in 3D. From the left to right, top to bottom, the plots show the evolution at times $t=0$, $0.10$, $0.30$, $0.525$, $0.55$, $0.75$.}
\label{fig:mcm 3d dumbbell}
\end{figure}


\subsection{Anisotropic mean curvature motion}
So far, the motion law considered may be more appropriately stated as isotropic mean curvature motion. In \cite{oberman2011aniso}, Oberman et at., present a method for anisotropic mean curvature flow: 
\begin{align}
        u_t = (\gamma(\omega) + \gamma''(\omega))\abs{\nabla u} \nabla \cdot \left( \frac{\nabla u}{\abs{\nabla u}} \right),
\label{aniso mcm}
\end{align}
where $\omega = \arctan(u_y/u_x)$, and 
\begin{align}
        \gamma(\omega) 
= \gamma_n(\omega) 
= \frac{1}{n^2+1}(n^2 + 1 - \sin(n\omega)), 
\quad\text{for } n = 0,2,4,8.
\label{aniso mcm gamma}
\end{align}
Under isotropic mean curvature motion, a simple, smooth closed contour in 2D has a circular limiting shape as it reduces to a point. Under \cref{aniso mcm,aniso mcm gamma}, the limiting shape will have $n$-fold rotational symmetry.

For our methods, the added factor of $\gamma(\omega) + \gamma''(\omega)$ presents no additional difficulty at all. Again, we can stabilize with $p\Delta u$, with $p=(1+(n^2-1)/(n^2+1))p_0$. Setting $n=4$, and on a periodic grid of size $256\times 256$, we take 500 time steps with a linearly stabilized ETDRK2 to produce the solution presented in \cref{fig:aniso mcm}. 
\begin{figure}
        \centering
\begin{minipage}{0.44\textwidth}
       \includegraphics[width=0.96\textwidth]{aniso0}
\end{minipage}
\begin{minipage}{0.44\textwidth}
       \includegraphics[width=0.96\textwidth]{aniso001}
\end{minipage}
\begin{minipage}{0.44\textwidth}
       \includegraphics[width=0.96\textwidth]{aniso006}
\end{minipage}
\begin{minipage}{0.44\textwidth}
       \includegraphics[width=0.96\textwidth]{aniso016}
\end{minipage}
\caption[Anisotropic mean curvature flow.]{Anisotropic mean curvature flow in 2D. The plots show the evolution of the curve at times $t=0$, $0.01$, $0.06$, $0.16$. The initial curve smooths and shrinks to a curve exhibiting four-fold symmetry as it collapses to a point.}
\label{fig:aniso mcm}
\end{figure}

\section{Phase Separation on Moo Surfaces}
For our last batch of experiments, we will define and evolve phase separation models on rectangular grids in 2D, and on surfaces in 3D using the closest point method \cite{ruuth2008simple,macdonald2009implicit}.

\subsection{Cahn-Hilliard on surface using the closest point method}
In this section, the model equation is 
\begin{align}
u_t = -\epsilon^2 \Delta_\S^2 u + \Delta_\S(u^3- u),
\label{ch}
\end{align}
where $\epsilon>0$ is a small parameter that determines interfacial thickness, $\S$ is a surface in $\bR^3$, and the solution $u$ indicates the domains of the separating binary fluid.

We assume initially that the solution is well-mixed and thus the initial conditions are chosen uniformly at random from $[-\epsilon,\epsilon]$.. As time progress, the initially well-mixed solution aggregates into homogeneous zones separated by transition layers of thickness $\O(\epsilon)$. Within the homogeneous zones, $u$ takes on the value 1 or $-1$. 

To discretize the surface operators, we use the closest point method of Ruuth and Merriman \cite{ruuth2008simple}. Their approach is to extend the solution off the surface into the embedding space at each time step in a way that allows them to replace the surface operators with its Cartesian analog. It then becomes a matter of selecting from standard finite difference methods for the discretization of the spatial operators. We must also mention the work of Macdonald and Ruuth \cite{macdonald2009implicit} that opened up the possibility of implicit time stepping with the closest point method and the open access closest point method software \cite{cpmcodes}.

To use a linearly stabilized method for time stepping, we must determine $p$, and in order to determine $p$, we will linearize \cref{ch}. Substituting with $u = u^n + \delta v$, we have 
\begin{align}
\begin{split} 
(u^n + \delta v)_t 
&= -\epsilon^2\Delta_\S^2 (u^n + \delta v) 
+ \Delta_\S((u^n + \delta v)^3 - u^n - \delta v) 
\\
&= -\epsilon^2\delta\Delta_\S^2 v
+ \delta^3\Delta_\S v^3 + 3\delta^2u^n \Delta_\S v^2 + 3\delta (u^n)^2 \Delta_\S v - \delta \Delta_\S v. 
\end{split} 
\label{ch linearize}
\end{align}
Differentiating with respect to $\delta$ and then setting $\delta = 0$, we arrive at the linearized equation 
\begin{align}
v_t = -\epsilon^2\Delta_\S^2 v + (3(u^n)^2-1) \Delta_\S v. 
\end{align}
Thus if we are to stabilize with $p\Delta_\S u$, then we may choose $p$ as 
\begin{align}
p \geq (3\bar u^2 - 1)p_0, 
\end{align}
where $\bar u = \sup_\S \abs{u^n}$. For simplicity, we use $p=2p_0$ as $u$ takes values in $[-1,1]$.  

\cref{fig:ch torus,fig:ch cow} show the solution to \cref{ch} on two different surfaces at time $T=100$. In both examples, the spatial grid is uniform, with $h=0.14$ for the torus and $h=0.075$ for the cow. We then set $\epsilon=2h$. Time stepping is done with SBDF2 and for the time step-size, we have chosen $\Delta t = 0.5h$ in the example with the torus and $\Delta t = 1$ in the example with the cow. 

The latter is to illustrate that we may choose the step-size reflecting the dynamics of the solution. For the Cahn-Hilliard, the dynamics may be broadly split into two stages \cite{sun2000dynamics}. The first is a fast coarsening stage in which a pattern of internal layers is formed from the initial state in an $\O(1)$ time interval. This is then followed by an exponentially slow coarsening during which the internal layers may collapse together to reach a stable/energy minimizing state. It is during this slow phase that our schemes provide the greatest benefit. 

\begin{figure}[htb!]
	\centering
\includegraphics[width=0.65\textwidth]{ch_torus}
\caption[Cahn-Hilliard on a torus.]{Cahn-Hilliard on a torus.}
\label{fig:ch torus}
\end{figure}
\begin{figure}[htb!]
	\centering
\includegraphics[width=0.65\textwidth]{ch_cow}
\caption[Cahn-Hilliard on a cow-shaped surface.]{Cahn-Hilliard on a cow-shaped surface.}
\label{fig:ch cow}
\end{figure}

\subsection{A modified Cahn-Hilliard with nonlocal interactions}
As our last example, we solve 
\begin{align}
        u_t = -\epsilon^2 \Delta^2 u + \Delta (u^3 + 3mu^2 - (1-3m^2)u) - u, 
\quad (x,y)\in[-2\pi,2\pi]^2,
\quad t>0,
\label{mch}
\end{align}
with periodic boundary conditions and uniform random perturbations in $[-\epsilon,\epsilon]$ for the initial condition. In this model, $\epsilon>0, m\geq 0$ are parameters that characterize the steady state behaviour of the solution \cite{choksi2009phase,choksi20112d}. As described in \cite{choksi2009phase}, \cref{mch} is of interest for its connections to the self-assembly of diblock copolymers. They examine how different regions in the parameter space $(\epsilon, m)$ dictate the formation of distinct energy-minimizing patterns from an initial well-mixed state. They also address the difficulties one expects when numerically generating these energy-minimizing states. As in the previous example, we propose our methods for handling the exponentially slowing dynamics that take place as the solution inches toward steady state. We will show that our methods are stable and can quickly advance the solution to access the minimizing states. 

To stabilize \cref{mch}, we first need an estimate on the term $\Delta(u^3 + 3mu^2-(1-3m^2)u)$. We proceed as in \cref{ch linearize} to find the linearized form
\begin{align}
v_t = -\epsilon^2\Delta^2 v - v + (3(u^n)^2 + 6mu^n - (1-3m^2) \Delta v.
\end{align}
Thus if we stabilize with $p\Delta u$, choosing $p \geq (2+6m+3m^2)p_0$ will be sufficient to guaranteeing unconditional stability.

The experiments in \cref{fig:mchp energy minimizers}  show that our methods can be used to access the energy-minimizing states of \cref{mch}. As a final note, we mention that we adopt the strategy of spectral filtering that was proposed in \cite{choksi20112d}.

\begin{figure}[htb!]
        \centering
\begin{minipage}{0.45\textwidth}
       \includegraphics[width=0.84\textwidth]{lamellae}
\end{minipage}
\begin{minipage}{0.45\textwidth}
       \includegraphics[width=0.84\textwidth]{hex_pack}
\end{minipage}
\caption[{Energy minimizing patterns of the nonlocal Cahn-Hilliard \cref{mch}.}]{Energy minimizing patterns of the nonlocal Cahn-Hilliard \cref{mch}. Left: Lamellae, with $(\epsilon,m)=(0.10,0)$. Right: Hexagonally packed spots, with $(\epsilon,m)=(0.10,0.40)$.}
\label{fig:mchp energy minimizers}
\end{figure}


\chapter{Conclusion}
In this thesis, we gave details of a framework for developing effective linearly stabilized time stepping methods. As was already known, unconditional stability is required. In our work, we further deduced that the parameter restriction over which we enjoy unconditional stability must be unbounded. 

Alongside stability, there is the matter of accuracy. As the former dictated the range from which we can select the parameter $p$, it was then natural to explore how the discretization error behaves as a function of $p$. We recommended the simple procedure of applying the numerical method to the modified test equation and examining the series expansion at $\Delta t = 0$. When the coefficient of the leading order error term is a degree two or higher polynomial in $p$, the schemes fared poorly for $p$ large.

We also addressed the feasibility of taking large time step-sizes. The schemes that perform well possess strong damping as $z\to-\infty$. We have shown that this quality, as was with the previous two, is easy to check for any time stepping method.

On the matter of developing linearly stabilized schemes with the aforementioned properties, we have proposed a number of new methods based on IMEX multistep methods and exponential Runge-Kutta methods that perform exceptionally on at least two of the three, but none that perform strongly in all three.

We recommend SBDF2 for its ease of use and its superior damping to CNAB, although CNAB remains a useful alternative as it has a smaller error constant as $\Delta t \to 0$. Implementing these schemes require no more expertise than applying an implicit method to solve a constant coefficient heat equation. 
When the domain is periodic and  $p$ can be chosen quite small, ETDRK2 and ETDRK4 offer strong performance at large step-sizes.
Of the existing linearly stabilized methods, SBDF1 and the EIN method, neither are competitive with our schemes due to some combination of slower convergence rate and comparatively high computing times.

A number of questions have been raised throughout this thesis and are worthy of further consideration. We have identified three properties critical for effective linearly stabilized schemes, and the derivation of high order methods satisfying all three remains open. While the metrics we supplied are easy to compute sufficient as a guide, a study of the discretization error on a fully nonlinear problem would elevate our current level of understanding. Adaptivity could also be investigated for both the step-size and for selecting $p$. Finally, comparing our methods to popular algorithms in applications such as image processing would be of interest.


%   BACK MATTER  %%%%%%%%%%%%%%%%%%%%%%%%%%%%%%%%%%%%%%%%%%%%%%%%%%%%%%%%%%%%%%
%
%   References and appendices. Appendices come after the bibliography and
%   should be in the order that they are referred to in the text.
%
%   If you include figures, etc. in an appendix, be sure to use
%
%       \caption[]{...}
%
%   to make sure they are not listed in the List of Figures.
%

\backmatter%
	\addtoToC{Bibliography}
	\bibliography{master_ref}

\begin{appendices} % optional
	\chapter{Code}
\end{appendices}
\end{document}
