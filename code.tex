\chapter{Code}
\section{Time integration of the Cahn-Hilliard equation using linearly stabilized time stepping schemes}
\label{A1 sample code}
\begin{verbatim}
function[]=CHp_2D()
% Solve the Cahn-Hilliard in 2d,
%   u_t = -ep^2*L^2(u) + L(u^3-u), (x,y)\in [0,2*pi]^2, t>0,
% with spectral method in space and a linearly stabilized method in time.
% For 3d, change wave numbers to the commented line. Adjust LR and the
% means in the ETDRK schemes. Adjust visualization commands.

%% Parameters: spatial, temporal, model, visualization
  N=256;  L=2*pi;
  T0=0; Tf=10; nt=400; dt=(Tf-T0)/nt; meth='cnab'; meth=upper(meth);
  epsilon=0.1; ep2=epsilon^2; 
  vis_update=40;

%% Initial condition and storage
  rng('default'); u=2*epsilon*(rand(N,N)-0.50); v=fftn(u);
  
%% System for inversion
% Wave numbers
  k=[0:N/2-1 0 -N/2+1:-1]'*(2*pi/L); k=-k.^2; 
  k=bsxfun(@plus,k,k'); 
%   k=bsxfun(@plus,bsxfun(@plus,k,k'),reshape(k,1,1,N));
  kdt=dt*k;
% Choose method
  switch meth
    case 'SBDF1'
      fignum=1; p=1.0; A=1-p*kdt+(dt*ep2)*k.^2;
      
    case 'SBDF2'
      fignum=2; p=1.5; A=1+(2/3)*(-p*kdt+(dt*ep2)*k.^2);
      
    case 'CNAB'
      fignum=3; p=2; A=1+(0.5)*(-p*kdt+(dt*ep2)*k.^2); 
      B=(1-0.5*(-p*kdt+(dt*ep2)*k.^2))./A;
      
    case 'ETDRK4'
      fignum=4; p=1.0; 
      Ln=p*kdt-(dt*ep2)*k.^2; E=exp(Ln); E2=exp(0.5*Ln);
      M=16; r=exp(1i*pi*((1:M)-0.5)/M);
      LR=bsxfun(@plus,Ln(:,:,ones(M,1)),reshape(r,1,1,M));
      A=kdt.*real(mean((exp(0.5*LR)-1)./LR,3));
      b1=kdt.*real(mean((-4-LR+exp(LR).*(4-3*LR+LR.^2))./LR.^3,3));
      b2=2*kdt.*real(mean((2+LR+exp(LR).*(-2+LR))./LR.^3,3));
      b4=kdt.*real(mean( (-4-3*LR-LR.^2+exp(LR).*(4-LR))./LR.^3,3));
  
    case 'ETDRK2'
      fignum=5; p=1.0;
      Ln=p*kdt-(dt*ep2)*k.^2; E=exp(Ln); 
      M=16; r=exp(1i*pi*((1:M)-0.5)/M);
      LR=bsxfun(@plus,Ln(:,:,ones(M,1)),reshape(r,1,1,M));
      A1=kdt.*real(mean((exp(LR)-1)./LR,3));
      A2=kdt.*real(mean((exp(LR)-LR-1)./LR.^2,3)); 
  end
  fg=figure(100+fignum);
  
%% Begin time stepping
  for kt=1:nt,
    switch meth
      case 'SBDF1'
        rhs=v+kdt.*fftn(u.^3-(1+p)*u); v=rhs./A;
        
      case 'SBDF2'
        if kt==1, 
          v1=v; f1=kdt.*fftn(u.^3-(1+p)*u);
          v=(v1+f1)./(1-p*kdt+(dt*ep2)*k.^2);
        else 
          f=kdt.*fftn(u.^3-(1+p)*u); 
          u=((4/3)*v-(1/3)*v1+(2/3)*(2*f-f1))./A; 
          v1=v; f1=f; v=u;
        end
        
      case 'CNAB'
        if kt==1,
          f1=kdt.*fftn(u.^3-(1+p)*u); 
          v=(v+f1)./(1-p*kdt+(dt*ep2)*k.^2);
        else
          f=kdt.*fftn(u.^3-(1+p)*u);
          v=B.*v+0.5*(3*f-f1)./A;
          f1=f; 
        end
        
      case 'ETDRK4'
        f=fftn(u.^3-(1+p)*u); a=E2.*v+A.*f; ar=real(ifftn(a));
        fa=fftn(ar.^3-(1+p)*ar); b=E2.*v+A.*fa; br=real(ifftn(b));
        fb=fftn(br.^3-(1+p)*br); c=E2.*a+A.*(2*fb-f); cr=real(ifftn(c));
        fc=fftn(cr.^3-(1+p)*cr);
        v=E.*v + b1.*f + b2.*(fa+fb) + b4.*fc;
        
      case 'ETDRK2'
        f=fftn(u.^3-(1+p)*u); a=E.*v+A1.*f; ar=real(ifftn(a));
        fa=fftn(ar.^3-(1+p)*ar); 
        v=a+A2.*(fa-f);
    end
    u=real(ifftn(v));
    
  % Plots
    if kt==nt || mod(kt,vis_update)==0,
      set(0,'currentfigure',fg); drawnow;
      pcolor(u); shading('interp'); axis('equal','off'); 
      title(['CH by ',meth,' with dt=',num2str(dt,'%10.2g'), ...
        ' at T=',num2str(dt*kt,'%10.2g')]); 
    end
  end
end
\end{verbatim}
